% arara: xelatex
% arara: makeglossaries
% arara: xelatex
% arara: xelatex

% style inspired by
% https://github.com/omelkonian/presentations/blob/2c0b2e1f592a8f90797e4997c3ab8785b114f595/%5B2019.08.20%5D%20BitML%20(SRC%20Presentation%20@%20ICFP)/bitml-presentation.tex

\documentclass[aspectratio=169]{beamer}
\usetheme{metropolis}

\usepackage[utf8]{inputenc}
\usepackage[T1]{fontenc}
\usepackage{textcomp}
\usepackage[english]{babel}
\usepackage{amsmath, amssymb, bm}
\usepackage{tcolorbox}
\usepackage[makeroom]{cancel}
\usepackage{mathpartir}
\usepackage{xparse}

% displayquote
\usepackage{csquotes}

% TikZ
\usepackage{tikz}

% figure support
\usepackage{import}
\usepackage{xifthen}
\usepackage{pdfpages}
\usepackage{transparent}
\newcommand{\incfig}[1]{%
	\def\svgwidth{\columnwidth}
	\import{./figures/}{#1.pdf_tex}
}

% style for tcolorboxes
\tcbset{plain/.style={colbacktitle=white,coltitle=black,colback=white}}
\pdfsuppresswarningpagegroup=1

% fonts
\usepackage{relsize}
\usepackage[tt=false]{libertine}
\usepackage[libertine]{newtxmath}

% colours
\definecolor{CTUBlue}{HTML}{175BB0}
\setbeamercolor{alerted text}{fg=CTUBlue}


\title{P4 Language Server}
\author{Bc. Ondřej Kvapil, supervised by Ing. Viktor Puš, Ph.D., MBA}
\institute{Czech Technical University in Prague, Faculty of Information Technology}
\date{14th June, 2023}


\begin{document}

\begin{center}
	\maketitle
\end{center}

%\begin{frame}{Outline}
%	\begin{itemize}
%		\item Introduction
%		\item Goals of the thesis
%		\item Motivation behind the choice of topic
%		\item Technical terminology and acronyms
%		\item Current state of the solution
%		\item Solution of the problem (expected outcome)
%		\item Conclusion -- a summary of the important points, contributions
%		\item Thanks \& discussion
%	\end{itemize}
%\end{frame}

\begin{frame}{Programming Protocol-independent Packet Processors}
	\begin{itemize}
		\item<1-> P4 is a language for \alert{software-defined networking}
		\item<2-> Specifies packet switching and routing operations
		\item<3-> Configures programmable network processors
	\end{itemize}
\end{frame}

\begin{frame}{Motivation}
	\begin{itemize}
		\item A fairly new language, first appeared in 2014 \pause
		\item \alert{Lacks tooling} available for general-purpose programming
			languages \pause
		\item Network engineers experience slow development cycles \pause
		\item Compilation process can take up to an hour \pause
		\item Diagnostics are limited to compiler errors and warnings
	\end{itemize}
\end{frame}

\begin{frame}{Goals}
	\begin{itemize}
		\item Equip network engineers with modern language tooling \pause
		\item Shorten the feedback cycle \pause
		\item Develop an open-source analyser for P4 \pause
		\item Integrate with the Visual Studio Code editor
	\end{itemize}
\end{frame}

\begin{frame}{Our solution}
	\begin{itemize}
		\item General-purpose \alert{incremental} packrat \alert{parsing}
		library \pause
		\item A domain-specific language for grammar construction \pause
		\item High-level macros for \alert{typed abstract syntax trees} \pause
		\item All based on an extensible incremental architecture
	\end{itemize}
\end{frame}

\begin{frame}{Our solution}
	\begin{overprint}
		\onslide<1>

		\onslide<2>
	\end{overprint}
\end{frame}

\begin{frame}{Results}
	\begin{itemize}
		\item Proof-of-concept language server for P4 \pause
		\item Integrated with Visual Studio Code \pause
		\item Plugs into other LSP-capable editors \pause
		\item Supports \alert{error-tolerant autocompletion} \pause
		\item Provides diagnostics for preprocessor issues \pause
		\item Features simple go-to definition functionality
	\end{itemize}
\end{frame}

\begin{frame}{Summary}
	\begin{itemize}
		\item LSP-compliant analyser for P4 \pause
		\item Incremental analysis pipeline \pause
		\item Packrat parsing
	\end{itemize}
\end{frame}

\begin{frame}{}
	Thank you for listening.

	\bibliographystyle{./styles/iso690.bst}
	\begin{thebibliography}{1}
		\bibitem{thesis}
		Kvapil, Ondřej.
		\textit{P4 Language Server}.
		Master's thesis.
		Czech Technical University in Prague,
		Faculty of Information Technology, 2021.
	\end{thebibliography}

	\url{https://github.com/viluon/masters-thesis}
\end{frame}

\begin{frame}{Questions for the defense}
	\alert{
		\begin{overprint}
			\onslide<1-2> Question 1
			\onslide<3-4> Question 2
			\onslide<5-6> Question 3
		\end{overprint}
	}

	\begin{displayquote}
		\begin{overprint}
			\onslide<1-2>
			Did you consider some alternative approaches for the incremental
			parsing? Popular incremental parser is \texttt{treesitter} (although
			I have not seen it used in any LSP I know).

			\onslide<3-4>
			I don't know how large the P4 programs tend to be, but did you look
			at how much the incrementality actually helps?

			\onslide<5-6>
			Did you consider using \texttt{rowan} for the lossless syntax trees?
		\end{overprint}
	\end{displayquote}

	\begin{overprint}
		\onslide<2>
		yada

		\onslide<4>
		tada

		\onslide<6>
		dada
	\end{overprint}
\end{frame}

\end{document}
