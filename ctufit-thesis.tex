%% This is the ctufit-thesis example file. It is used to produce theses
%% for submission to Czech Technical University, Faculty of Information Technology.
%%
%% Get the newest version from
%% https://gitlab.fit.cvut.cz/theses-templates/FITthesis-LaTeX
%%
%%
%% Copyright 2021, Eliska Sestakova and Ondrej Guth
%%
%% This work may be distributed and/or modified under the
%% conditions of the LaTeX Project Public Licenese, either version 1.3
%% of this license or (at your option) any later version.
%% The latest version of this license is in
%%  https://www.latex-project.org/lppl.txt
%% and version 1.3 or later is part of all distributions of LaTeX
%% version 2005/12/01 or later.
%%
%% This work has the LPPL maintenance status `maintained'.
%%
%% The current maintainer of this work is Ondrej Guth.
%% Contact ondrej.guth@fit.cvut.cz for bug reports.
%% Alternatively, submit bug reports into the tracker at
%% https://gitlab.fit.cvut.cz/theses-templates/FITthesis-LaTeX/issues
%%
%%

%%%%%%%%%%%%%%%%%%%%%%%%%%%%%%%%%%%%%%%%%
% CLASS OPTIONS
% language: czech/english/slovak
% thesis type: bachelor/master/dissertation
%%%%%%%%%%%%%%%%%%%%%%%%%%%%%%%%%%%%%%%%%
\documentclass[english,master,unicode]{ctufit-thesis}

%%%%%%%%%%%%%%%%%%%%%%%%%%%%%%%%%%
% FILL IN THIS INFORMATION
%%%%%%%%%%%%%%%%%%%%%%%%%%%%%%%%%%
\ctufittitle{P4 Language Server} % replace with the title of your thesis
\ctufitauthorfull{Bc. Ondřej Kvapil} % replace with your full name (first name(s) and then family name(s) / surname(s)) including academic degrees
\ctufitauthorsurnames{Kvapil} % replace with your surname(s) / family name(s)
\ctufitauthorgivennames{Ondřej} % replace with your first name(s) / given name(s)
\ctufitsupervisor{Ing.\,Viktor Puš,\,Ph.D.\,MBA} % replace with name of your supervisor/advisor (include academic degrees)
\ctufitdepartment{Katedra teoretické informatiky} % replace with the department of your defence
\ctufityear{2023} % replace with the year of your defence
\ctufitdeclarationplace{Prague} % replace with the place where you sign the declaration
\ctufitdeclarationdate{\DTMdisplaydate{2023}{5}{4}{0}} % replace with the date of signature of the declaration

\ctufitabstractCZE{Jazyk P4 je používán pro konfiguraci programovatelných
síťových procesorů. Navzdory své popularitě v odvětví Software Defined
Networking ale zaostává co se podpory programátora týče. V této práci navrhujeme
a implementujeme language server pro jazyk P4, který poskytuje podporu pro
lexikální analýzu, preprocessing a syntaktickou analýzu. Na těchto základech
stavíme automatické doplňování, reporting chyb a navigaci ve zdrojovém kódu.
Language server je implementován v jazyce Rust a integrován do vývojového
prostředí Visual Studio Code.}

\ctufitabstractENG{The P4 language is used for configuring programmable network
processors. Despite its popularity in the Software Defined Networking field, it
suffers from a lack of modern developer tooling. In this thesis, we design and
implement a language server for the P4 language, which provides support for
lexical analysis, preprocessing, and parsing. On these foundations, we build
support for autocompletion, error reporting, and navigation in the source code.
The language server is implemented in the Rust language and integrated into the
Visual Studio Code development environment.}

\ctufitkeywordsCZE{language server protocol, language server, syntaktická analýza, sémantická analýza, P4, SDN, vývojářské nástroje}
\ctufitkeywordsENG{language server protocol, language server, parsing, semantic analysis, P4, SDN, developer tools}
%%%%%%%%%%%%%%%%%%%%%%%%%%%%%%%%%%
% END FILL IN
%%%%%%%%%%%%%%%%%%%%%%%%%%%%%%%%%%

%%%%%%%%%%%%%%%%%%%%%%%%%%%%%%%%%%
% CUSTOMIZATION of this template
% Skip this part or alter it if you know what you are doing.
%%%%%%%%%%%%%%%%%%%%%%%%%%%%%%%%%%

\RequirePackage{iftex}[2020/03/06]
\iftutex % XeLaTeX and LuaLaTeX
    \RequirePackage{ellipsis}[2020/05/22] %ellipsis workaround for XeLaTeX
\else
    \RequirePackage[utf8]{inputenc}[2018/08/11] %this file encoding
    \RequirePackage{lmodern}[2009/10/30] % vector flavor of Computer Modern font
\fi

% hyperlinks
\RequirePackage[pdfpagelayout=TwoPageRight,colorlinks=false,allcolors=decoration,pdfborder={0 0 0.1}]{hyperref}[2020-05-15]

% uncomment the following to hide all hyperlinks
% \RequirePackage[pdfpagelayout=TwoPageRight,hidelinks]{hyperref}[2020-05-15]

\RequirePackage{pdfpages}[2020/01/28]

\setcounter{secnumdepth}{4} % numbering sections; 4: subsubsection



%%%%%%%%%%%%%%%%%%%%%%%%%%%%%%%%%%
% CUSTOMIZATION of this template END
%%%%%%%%%%%%%%%%%%%%%%%%%%%%%%%%%%


%%%%%%%%%%%%%%%%%%%%%%
% DEMO CONTENTS SETTINGS
% You may choose to modify this part.
%%%%%%%%%%%%%%%%%%%%%%
\usepackage{dirtree}
\usepackage[style=iso]{datetime2}
\usepackage{fnpct}
\usepackage{lipsum,tikz}
\usepackage{csquotes}
\usepackage[acronym,nonumberlist,toc,numberedsection=autolabel]{glossaries}
\makeglossaries
\usepackage[colorinlistoftodos,prependcaption,textsize=tiny]{todonotes}
\usepackage[style=iso-numeric]{biblatex}
\addbibresource{text/bib-database.bib}
\usepackage{hyperref}
\usepackage{listings} % typesetting of sources
% \usepackage{minted} % typesetting of sources
\usepackage{multicol}
\usepackage{pdfcomment}
\usepackage{tikz}
\usetikzlibrary{automata,backgrounds,fit,positioning,shapes}
\usepackage{tcolorbox}
% libertine without tt
\usepackage[tt=false]{libertine}
\usepackage[libertine]{newtxmath}

% TODO: this is only for correct typesetting of todonotes
\setlength{\marginparwidth}{3.3cm}

%theorems, definitions, etc.
\theoremstyle{plain}
\newtheorem{theorem}{Theorem}
\newtheorem{lemma}[theorem]{Lemma}
\newtheorem{corollary}[theorem]{Corollary}
\newtheorem{proposition}[theorem]{Proposition}
\newtheorem{definition}[theorem]{Definition}
\theoremstyle{definition}
\newtheorem{example}[theorem]{Example}
\theoremstyle{remark}
\newtheorem{note}[theorem]{Note}
\newtheorem*{note*}{Note}
\newtheorem{remark}[theorem]{Remark}
\newtheorem*{remark*}{Remark}
\numberwithin{theorem}{chapter}
%theorems, definitions, etc. END
%%%%%%%%%%%%%%%%%%%%%%
% DEMO CONTENTS SETTINGS END
%%%%%%%%%%%%%%%%%%%%%%


% locate combining diacritics
\DeclareUnicodeCharacter{0301}{*************************************}

\begin{document}
\frontmatter\frontmatterinit % do not remove these two commands

\includepdf{resources/assignment-include.pdf} % replace that file with your thesis assignment provided by study office

\thispagestyle{empty}\cleardoublepage\maketitle % do not remove these three commands

\imprintpage % do not remove this command

\tableofcontents % do not remove this command
%%%%%%%%%%%%%%%%%%%%%%
% list of other contents: figures, tables, code listings, algorithms, etc.
% add/remove commands accordingly
%%%%%%%%%%%%%%%%%%%%%%
\listoffigures % list of figures
\begingroup
\let\clearpage\relax
\listoftables % list of tables

\endgroup % Moved the \endgroup here because list of listings had heading on one page and the list on the next

\lstlistoflistings % list of source code listings generated by the listings package
% \listoflistings % list of source code listings generated by the minted package
%%%%%%%%%%%%%%%%%%%%%%
% list of other contents END
%%%%%%%%%%%%%%%%%%%%%%

%%%%%%%%%%%%%%%%%%%
% ACKNOWLEDGMENT
% FILL IN / MODIFY
% This is a place to thank people for helping you. It is common to thank your supervisor.
%%%%%%%%%%%%%%%%%%%
\begin{acknowledgmentpage}
	I would like to thank Timothy Roberts and Adam Reynolds for their kindness
	and support of my internship during challenging times at Intel. Thanks to my
	supervisor, Viktor Puš, without whose enthusiasm I would never have applied
	for the internship. And, last but not least, thanks to my family and friends
	for making my life better than I deserve.
\end{acknowledgmentpage}
%%%%%%%%%%%%%%%%%%%
% ACKNOWLEDGMENT END
%%%%%%%%%%%%%%%%%%%


%%%%%%%%%%%%%%%%%%%
% DECLARATION
% FILL IN / MODIFY
%%%%%%%%%%%%%%%%%%%
% INSTRUCTIONS
% ENG: choose one of approved texts of the declaration. DO NOT CREATE YOUR OWN. Find the approved texts at https://courses.fit.cvut.cz/SFE/download/index.html#_documents (document Declaration for FT in English)
% CZE/SLO: Vyberte jedno z fakultou schvalenych prohlaseni. NEVKLADEJTE VLASTNI TEXT. Schvalena prohlaseni najdete zde: https://courses.fit.cvut.cz/SZZ/dokumenty/index.html#_dokumenty (prohlášení do ZP)
\begin{declarationpage}
	I hereby declare that the presented thesis is my own work and that I have
	cited all sources of information in accordance with the Guideline for
	adhering to ethical principles when elaborating an academic final thesis.

	I acknowledge that my thesis is subject to the rights and obligations
	stipulated by the Act No. 121/2000 Coll., the Copyright Act, as amended. In
	accordance with Section 2373(2) of Act No. 89/2012 Coll., the Civil Code, as
	amended, I hereby grant a non-exclusive authorization (licence) to utilize
	this thesis, including all computer programs that are part of it or attached
	to it and all documentation thereof (hereinafter collectively referred to as
	the "Work"), to any and all persons who wish to use the Work. Such persons
	are entitled to use the Work in any manner that does not diminish the value
	of the Work and for any purpose (including use for profit). This
	authorisation is unlimited in time, territory and quantity.
\end{declarationpage}
%%%%%%%%%%%%%%%%%%%
% DECLARATION END
%%%%%%%%%%%%%%%%%%%

\printabstractpage % do not remove this command

%%%%%%%%%%%%%%%%%%%
% SUMMARY
% FILL IN / MODIFY
% OR REMOVE ENTIRELY (upon agreement with your supervisor)
% (appropriate to remove in most theses)
%%%%%%%%%%%%%%%%%%%
\begin{summarypage}
\section*{Introduction}

We open with an overview of the networking setting that led to the development
of the P4 language. After a brief survey of existing tooling, we decide to
implement our own.

\section*{The P4 Language}

Next, we delve into the details of P4 to better understand the commonalities and
differences between it and conventional programming languages.

\section*{Language Server Architecture}

We explore the language server protocol and existing LSP-compliant tools. We
also survey architectural themes in conventional compilers and pluck a few ripe
ideas for our own use later.

\section*{Design}

In this chapter, we detail the design of our language server. We go over the key
abstractions and algorithms, motivating our design decisions along the way.

\section*{Results}

The final chapter closes with lessons learned, our implementation's strengths,
and areas where it falls short. We conclude with a discussion of future work.

\end{summarypage}
%%%%%%%%%%%%%%%%%%%
% SUMMARY END
%%%%%%%%%%%%%%%%%%%

%%%%%%%%%%%%%%%%%%%
% ABBREVIATIONS
% FILL IN / MODIFY
% OR REMOVE ENTIRELY
% List the abbreviations in lexicography order.
%%%%%%%%%%%%%%%%%%%

\printglossaries
%%%%%%%%%%%%%%%%%%%
% ABBREVIATIONS END
%%%%%%%%%%%%%%%%%%%

\mainmatter\mainmatterinit % do not remove these two commands

%%%%%%%%%%%%%%%%%%%
% THE THESIS
% MODIFY ANYTHING BELOW THIS LINE
%%%%%%%%%%%%%%%%%%%

\setcounter{page}{1}

% acronyms
\newacronym{api}{API}{application programming interface}
\newacronym{asic}{ASIC}{application-specific integrated circuit}
\newacronym{ast}{AST}{abstract syntax tree}
\newacronym{cfg}{CFG}{context-free grammar}
\newacronym{dsl}{DSL}{domain-specific language}
\newacronym{fpga}{FPGA}{field-programmable gate array}
\newacronym{fsm}{FSM}{finite state machine}
\newacronym{ide}{IDE}{integrated development environment}
\newacronym{ir}{IR}{intermediate representation}
\newacronym{llvm}{LLVM}{Low-Level Virtual Machine}
\newacronym{lsp}{LSP}{Language Server Protocol}
\newacronym{lto}{LTO}{link-time optimization}
\newacronym{onf}{ONF}{Open Networking Foundation}
\newacronym{p4}{P4}{Programming Protocol-independent Packet Processors}
\newacronym{peg}{PEG}{parsing expression grammar}
\newacronym{repl}{REPL}{read-eval-print loop}
\newacronym{sdn}{SDN}{software-defined networking}
\newacronym{xml}{XML}{eXtensible Markup Language}
\newacronym{yacc}{YACC}{Yet Another Compiler-Compiler}
\newcommand{\pfs}{\texorpdfstring{{P}4\textsubscript{16}}{P4 16}}
\newcommand{\pdfacrshort}[1]{\texorpdfstring{\acrshort{#1}}{\glsentryshort{#1}}}
\newcommand{\extern}{\texttt{extern}}
\newcommand{\citelink}[2]{\hyperlink{cite.\therefsection @#1}{#2}}
\newcommand{\linkedciteauthor}[1]{\citelink{#1}{\Citeauthor*{#1}}}

\lstdefinelanguage{p4}{
	keywords={abstract,action,apply,bit,bool,const,control,default,else,enum,
	error,extern,exit,false,header,header_union,if,in,inout,int,match_kind,
	package,parser,out,return,select,state,struct,switch,table,this,transition,
	true,tuple,type,typedef,value_set,varbit,verify,void},
	sensitive=true,
	morecomment=[l]{//},
	morecomment=[s]{/*}{*/},
	morestring=[b]",
}

% Do not forget to include Introduction
%---------------------------------------------------------------
% \chapter{Introduction}
% uncomment the following line to create an unnumbered chapter
\chapter*{Introduction}\addcontentsline{toc}{chapter}{Introduction}\markboth{Introduction}{Introduction}
%---------------------------------------------------------------

% The following environment can be used as a mini-introduction for a chapter.
% Use that any way it pleases you (or comment it out). It can contain, for
% instance, a summary of the chapter. Or, there can be a quotation.
\begin{chapterabstract}
	\dots in which we get to know the context that gave rise to the
	\acrshort{p4} language and the challenges we set out to overcome.
\end{chapterabstract}

\acrfull{p4} is a domain-specific language for programming network switches. Its
release started a shift in the field of \acrfull{sdn} which, up to that point,
relied heavily on fixed-function hardware for high-performance networking
applications. Similarly to how the C language became a de facto portable
assembler, abstracting over the details of each microprocessor, \acrshort{p4}
abstracts over the details of network processors by presenting a deeply
customizable interface shared by networking software and hardware alike.

Unlike previous approaches in \acrshort{sdn}, \acrshort{p4} does not have
built-in support for common network protocols like TCP, IP, or Ethernet.
Instead, it provides protocol-independent constructs that users can leverage in
order to define arbitrary protocols and instruct a flexible network switch on
how to handle them. This programmability lets a network engineer specify the
configuration and packet processing steps of a router architecture independently
of the underlying machinery.

\todo[inline]{we need a data/control plane description/intro}
% data plane\footnote{Also known as the ``forwarding plane."}
% A \acrshort{p4} compiler
% can then synthesize microcode for a given platform and generate library files
% for the control plane.

The newfound flexibility of network processor programming that \acrshort{p4}
enables does not get a chance to shine on traditional network hardware, which is
built for a predetermined set of protocols and processing functions. The real
power of \acrshort{p4} is unlocked by \emph{programmable} network processors,
which can be reconfigured to support novel protocols and forwarding setups. The
commercial sector answered the call for such hardware, for example in Intel's
Tofino line of
chips\footnote{\url{https://www.intel.com/content/www/us/en/products/network-io/programmable-ethernet-switch.html}}.

\acrshort{p4} became wildly popular in \acrshort{sdn} since its introduction in
2014\cite{p4original}, sparking both research and commercial applications. Three
years later, \acrshort{p4} underwent a major redesign\cite{p416}, which
simplified the syntax and removed special-purpose language constructs in favour
of more general solutions\footnote{Specifically, features like counters and
checksum units were replaced by \extern{}s, a universal construct for specifying
additional hardware capabilities not explicitly covered by the core syntax. The
language shrank from over 70 to less than 40
keywords\cite{p416:v1:spec:comparison}.}. The redesigned language is known as
\pfs\cite{p416:v123:spec} and its specification has since received more
incremental updates.

%-------------------------------%
\section*{A Growing Community}
%-------------------------------%

\acrshort{p4} gave rise to a varied ecosystem of commercial offerings of both
hardware and software. It spawned several academic projects that investigated
its semantics\cite{doenges2021petr4}, allocation to heterogeneous
hardware\cite{sultana2021flightplan}, open-source network
testing\cite{antichi2014osnt}, and sketch-based
monitoring\cite{namkung2022sketchlib}, among many
others\cite{liatifis2023advancingp4survey}. The \acrfull{onf}
maintains\cite{p4onf} an open-source reference implementation of a
\pfs\footnote{Although primarily developed for \pfs since the revision of the
language, it is also capable of migrating P4\textsubscript{14} programs to \pfs
or directly compiling them.} compiler frontend and mid\-end, along with several
backends:

\begin{description}
	\item[p4c-bm2-ss] Targets a sample software switch for testing purposes.
	\item[p4c-dpdk] Targets the DPDK software switch (SWX) pipeline\cite{dpdkDPDKRelease}.
	\item[p4c-ebpf] Generates C code which can be compiled to eBPF
		and then loaded in the Linux kernel.
	\item[p4test] A source-to-source P4 translator for testing, learning compiler
		internals and debugging.
	\item[p4c-graphs] Generates visual representations of P4 programs.
	\item[p4c-ubpf] Generates eBPF code that runs in user-space.
	\item[p4tools] A platform for P4 test utilities, includes a test-case
		generator for P4 programs.
\end{description}

All of these components are open source. The frontend and mid\-end of the
reference compiler provide a foundation for hardware vendors to support
\acrshort{p4} in their products and serve the community in resolving
discrepancies between commercial compilers and the language specification.

Despite this growth, \acrshort{p4} has only mediocre support for real-time
feedback to the programmer -- the vast majority of open and commercial tools
rely on compiler output to provide semantic insight into a P4
program\cite{p4insight}, or, as evidenced by the open backends, are parts of the
compiler proper. This is a problem, because the compiler was not designed for
interactive use. Compilations of complex programs can take over an hour to
complete. Long feedback loops hamper development. What's more, information
provided by the compiler frontend is very limited -- the compiler typically
reports at most one error message with very little (if any) explanation. The
lack of an integration between the compiler and development environments is an
obstacle to new users and a neglected area of \acrshort{p4} developer
experience. Even an ergonomic presentation of error and warning messages would
be a large improvement.

We can do far better, as interactive editing support for conventional
programming languages clearly demonstrates. Integrated developer environments
provide many features that \acrshort{p4} programmers can only dream of,
including autocompletion, documentation pop-ups, real-time diagnostics, code
navigation, automatic formatting, and refactoring. These features are not only
convenient, but also improve the quality of software by reducing the cognitive
load on the programmer.

%-----------------------------------------%
\section*{Language Servers to the Rescue}
%-----------------------------------------%

Over the last decade, \emph{language servers} became a popular
architecture\cite{barros2022editing} for providing semantically informed editing
features in integrated developer environments and lightweight source code
editors alike. Although examples of language server -like tools predate their
standardization\cite{bour2018merlin}, a major milestone in their development was
the introduction of the \acrfull{lsp}. The support of \acrshort{lsp} in Visual
Studio Code seeded an ever-growing environment of cross-platform tooling for a
variety of programming, configuration, specification, and markup languages.
Their success can be attributed in part to the investment by Microsoft. However,
\acrshort{lsp} as a standard has much technical merit.

Before the introduction of a common communication interface between
language-specific tooling and a given source code editor, implementors had to
create and maintain extensions for any number of different environments, often
in several programming languages. For example, take an editor plugin providing
smart completion and "go to definition" features for the Python programming
language. Suppose the author aims to extend Vim, Emacs, and IntelliJ IDEA. They
would therefore have to reimplement the given functionality in vimscript, Emacs
Lisp, and a JVM-based language. Even though recent innovations in classical text
editors\footnote{One such project is a fork and refactoring of Vi IMproved,
\emph{Neovim}: \url{https://github.com/neovim/neovim}.} have partly moved away
from using their own \acrshort{dsl}'s, the author is still burdened with three
times the implementation work, as every development environment \acrshort{api}
is different.

The solution to this problem is to split editor extensions according to the
client-server model. The major implementation work for domain-specific
functionality resides on the server side, whereas the client is only a
relatively thin wrapper that adapts a given editor's API. Implementors can build
many thin clients for different development environments that all connect to the
same \emph{language server}. \acrshort{lsp}'s raison d'être is support of this
model in Visual Studio Code.

Coden\-vy, Microsoft, and Red Hat collaborated on standardizing the protocol to
enable other tools to benefit from shared
tooling\cite{sdtimesCodenvyMicrosoft,infoworldMicrosoftbackedLanguage}.

% these comments make the section visible from the editor minimap
%-------------------%
%-------------------%
%-------------------%
%-------------------%
%-------------------%
%-------------------%
%-------------------%
%-------------------%
%-------------------%
%-------------------%
%-------------------%
\section{Outline}
%-------------------%
%-------------------%
%-------------------%
%-------------------%
%-------------------%
%-------------------%
%-------------------%
%-------------------%
%-------------------%
%-------------------%
%-------------------%
%-------------------%

\begin{enumerate}
	\item (intro) P4 information
	\item (intro) language servers, VS Code (screenshots, c++?)
	\item design: overview of the architecture, interfaces with the editor, what
		needs to be implemented
	\item subsections about problems that needed solving, but most should go
		into the design part!
	\item results, what it can do, open-source repository, performance metrics
		(aim for interactivity)
	\item conclusion, next steps
\end{enumerate}








\todo[inline]{authorship: link to GitHub, detail which non-critical parts were
implemented by a colleague}


\begin{lstlisting}[
	caption={~Useless code},
	label=list:8-6,
	captionpos=t,
	float,
	abovecaptionskip=-\medskipamount,
	belowcaptionskip=\medskipamount,
	language=C
]
	#include<stdio.h>
	#include<iostream>
	// A comment
	int main(void)
	{
		printf("Hello World\n");
		return 0;
	}
\end{lstlisting}

%%%%%%%%%%%%%%%%%%%%%%%%%%%%%%%%%
% alternative using package minted for source highlighting
% package minted requires execution with `-shell-escape'
% e.g., `xelatex -shell-escape ctufit-thesis.tex'
% \begin{listing}
% \caption{Zbytečný kód}\label{list:8-6}
% \begin{minted}{C}
%     #include<stdio.h>
%     #include<iostream>
%     // A comment
%     int main(void)
%     {
%         printf("Hello World\n");
%         return 0;
%     }
% \end{minted}
% \end{listing}
% %%%%%%%%%%%%%%%%%%%%%%%%%%%%%%%%%

\begin{table}\centering
\caption[Příklad tabulky]{~Typesetting math}\label{tab:mathematics}
\begin{tabular}{l|l|c|c}
	Typ	& Prostředí	&
		\LaTeX{}ovská zkratka	& \TeX{}ovská zkratka	\tabularnewline \hline
	Text	& \verb|math|	&
		\verb|\(...\)|	& \verb|$...$|	\tabularnewline \hline
	Displayed	& \verb|displaymath|	&
		\verb|\[...\]|	& \verb|$$...$$|	\tabularnewline
\end{tabular}
\end{table}


%---------------------------------------------------------------
\chapter{The P4 Language}
%---------------------------------------------------------------

\begin{chapterabstract}
	\dots in which we delve into the syntax and semantics of \acrshort{p4},
	explain its use cases, and discuss the differences to conventional programming
	languages.
\end{chapterabstract}

One of the original motivations behind \acrshort{p4} was the need for
restriction. While other \acrlong{dsl}s, such as Click\todo{citation}, existed
at the time, their embeddings within general-purpose programming languages made
it difficult to analyze data dependencies crucial for scheduling parallel
execution. The expressiveness of a language complicates its efficient
compilation. In the words of Robert Harper,

\begin{displayquote}
	\textit{The expressive power of a programming language arises from its
	strictures and \emph{not} from its affordances.}

	-- Robert Harper, \citedate*{pfpl1oplss2019} \cite{pfpl1oplss2019}
\end{displayquote}

Rather than embedding \acrshort{p4} in an existing language, these
considerations motivated a clean-slate design. However, to understand why is
packet processing any different from tasks suited to general-purpose programming
languages, we need to familiarise ourselves with the switching architecture.

%----------------------------%
\section{What's in a switch}
%----------------------------%

In the following text, we choose to use \emph{switch} to mean a general packet
forwarding device. Examples of switches are

A network switch
\todo[inline]{need a readable description of this \& network routing, also
citations in the text below}

Traditionally, these devices were examples of fixed-function hardware. Various
networking protocols were implemented directly in circuitry, which made them
efficient, but inflexible. It is impossible to reconfigure a fixed-function
\acrshort{asic} to process a protocol it was not explicitly designed for in
advance. If a new, backward-incompatible version of a given protocol emerges, or
if a hardware error is found in the chip, the network administrators need to
perform a costly hardware replacement in order to support, respectively
circumvent it. Moreover, fixed-function hardware design is a lengthy and
resource-intensive process. It may take several years before an updated
fixed-function chip hits the market.

The innovation of recent years is the introduction of \emph{programmable}
network processors, which can change the set of supported protocols on the fly.
These are similar to \acrshort{fpga}s in their reconfigurability, but
specialised to packet switching and routing, which makes them more efficient. A
programmable network switch has typically no prior knowledge of networking
protocols but contains efficient circuitry for parsing and pattern-matching in
order to support arbitrary\footnote{To some degree of complexity supported by
the circuit.} protocols uploaded to the chip as microcode. While programmable
networking hardware does incur a penalty for reconfigurability, when it comes to
efficiency, it sits between fixed-function devices and completely general-purpose
solutions.

Other implementations of packet switching are also common. The already mentioned
\acrlong{fpga}s can be programmed to simulate programmable or fixed-function
networking hardware, and thus allow an even higher degree of flexibility.
Naturally, \acrshort{fpga}s are less efficient than the circuits they simulate.
Finally, there are software switches, programs for widespread processor
architectures and operating systems, such as x86 and Linux. A software switch
represents the peak of flexibility, programmability, and requires no special
hardware other than what is already commonly present in conventional computers.
Purely software-based solutions cannot compete in energy efficiency with any of
the other approaches, but are often useful for testing and in small-scale
networks.

\begin{figure}[t]
	\includegraphics[
		trim={2.0cm 4.3cm 4.0cm 3.3cm}, % left bottom right top (insane, I know)
		width=1.00\textwidth
	]{resources/abstract-forwarding-model.png}
	\caption{The original \acrshort{p4}\textsubscript{14} abstract forwarding
	model, taken from \cite{p4original}.}
	\label{fig:abstract-forwarding-model}
\end{figure}

To support network configurations regardless of their physical implementation,
the original \acrshort{p4} paper defines the target-independent \emph{abstract
forwarding model}, outlined in Figure~\ref{fig:abstract-forwarding-model}. This
model assumes an end-to-end pipeline split into \emph{ingress} and \emph{egress}
parts. An arriving packet is first parsed to recognize the headers present
therein. These headers then travel through the pipeline's \emph{match-action
units}. A match-action unit performs limited pattern-matching and rewriting on
the parsed packet header. This part of the pipeline is partially configured at
runtime by the control plane. Forwarding rules defined in software are uploaded
to the network device.

\todo[inline]{all this needs a much better explanation! it's crucial for
understanding the P4 way of doing things.}

The model assumes that the payload is handled separately by the device and is
not available for pattern-matching.

\todo[inline]{finish the model description, mention deparsers, explain metadata}

A lifetime may end early if the packet is dropped during processing. Dropping is
indicated by mutating metadata.

%--------------------------%
\section{The \pfs language}
%--------------------------%

\begin{figure}[t]
	\includegraphics[width=1.00\textwidth]{resources/p4_16-architecture-model.png}

	\caption{\pfs program interfaces for an abstract architecture with two
	programmable blocks, taken from \cite{p416:v123:spec}.}
	\label{fig:arch-model}
\end{figure}

\acrshort{p4} is not a programming language for von Neumann
architectures\todo{starting with a negative example is poor taste}. The abstract
model assumes the target machine to be some sort of network processor with
programmable blocks embedded in a static pipeline. Figure~\ref{fig:arch-model}
illustrates such a machine and highlights the interfaces of the \acrshort{p4}
program.

A \acrshort{p4} program specifies a mapping of vectors of bits -- a bit\-vector
endomorphism. Every \acrshort{p4} program terminates; the language has no
looping constructs and no recursion, a compiler can thus determine the precise
maximum runtime of a program statically.

\todo[inline]{in discussing the subpar nature of the spec, include examples of
	mistakes.
	\href{https://p4.org/p4-spec/docs/P4-16-v-1.2.3.html\#sec-minsizeinbits}
	{compile-time size determination} mentions ``\emph{Each of these method
	calls evaluate to compile-time known values that return the minimum size in
	bits required to store the expression,}'' clearly forgetting to take the
	\texttt{maxSize*} functions into account}

\subsection{Syntax}

\todo[inline]{
	describe headers and their relation to structs, explain annotations
}

% relevant p4 spec sections

% 6.1.   Syntax and semantics
% 6.1.1. Grammar
% 6.1.2. Semantics and the P4 abstract machines
% 6.2.   Preprocessing
% 6.2.1. P4 core library
% 6.3.   Lexical constructs
% 6.3.1. Identifiers
% 6.3.2. Comments
% 6.3.3. Literal constants
% 6.4.   Naming conventions
% 6.5.   P4 programs
% 6.5.1. Scopes
% 6.5.2. Stateful elements
% 6.6.   L-values
% 6.7.   Calling convention: call by copy in/copy out
% 6.7.1. Justification
% 6.7.2. Optional parameters
% 6.8.   Name resolution
% 6.9.   Visibility

\pfs syntax is reminiscent of imperative programming languages in the C family.
It uses prefix notation for typed bindings, braces for lexical scoping blocks,
and semicolons to separate statements. However, instead of general-purpose
procedures, the top level constructs for executable code are mainly controls,
parsers, and actions. These will be explained in detail in this section.

\subsubsection*{\texttt{extern} objects and functions}

Before diving into the bulk of the syntactical forms that dominate user-written
\acrshort{p4} code, we need to introduce the general concept of
\texttt{extern}s. These special objects and functions were already present in
\acrshort{p4}\textsubscript{14} version 1.1. \pfs fully embraced these
constructs, which helped to both simplify and generalize the language.

\texttt{extern} objects and functions describe interfaces to facilities provided
by the architecture, as well as certain built-in language
constructs\todo{reference the parser declaration grammar section}. For example,
the \texttt{extern} object in Listing~\ref{lst:p4-extern-cksum} allows
\acrshort{p4} code to utilize a fixed-function checksum unit provided by the
target. The object specifies no implementation for the listed constructors and
methods.

To invoke a method of the checksum unit, the user needs first to
\emph{instantiate} the \texttt{extern} object. This syntactic form is shared
among all types with constructors: \texttt{control} blocks, \texttt{parser}s,
and \texttt{package}s\todo{describe those somewhere}. The effect of an
instantiation is to allocate the corresponding object, binding it to the
specified name.

The compiler is in charge of mapping instances of \texttt{extern}s to the target
architecture. If it does not find a mapping, either because the target does not
support the \texttt{extern} or does not have the resources to fit all its
instances, the compilation fails with an error.

\begin{lstlisting}[
	caption={~An \texttt{extern} object specifying the interface to the target's checksum unit.},
	label=lst:p4-extern-cksum,
	captionpos=t,
	tabsize=4,
	float,
	abovecaptionskip=-\medskipamount,
	belowcaptionskip=\medskipamount,
	language=p4
]
extern Checksum16 {
	Checksum16();              // constructor
	void clear();              // prepare unit for computation
	void update<T>(in T data); // add data to checksum
	void remove<T>(in T data); // remove data from existing checksum
	bit<16> get(); // get the checksum for the data added since last clear
}
\end{lstlisting}

\subsubsection*{Functions and parametrized code}

Functions in \acrshort{p4} come in several flavours. The first kind is known as
\emph{function declarations} and is perhaps the closest relative of procedures
in conventional imperative programming languages. The major difference is that
all parameters to a \acrshort{p4} function need to include a \emph{direction}.

\begin{description}
	\item[parameter direction] is either \texttt{in}, \texttt{out}, or
	\texttt{inout}. \texttt{in} indicates a read-only parameter with a defined
	value, \texttt{out} indicates a read-write parameter whose value is
	initially undefined upon entering the function body, and \texttt{inout}
	indicates a read-write parameter with a defined value.
\end{description}

Only l-values\todo{what are those?} can be passed as \texttt{out} and
\texttt{inout} parameters. The execution of the function call can change the
value of an \texttt{out} or \texttt{inout} parameter's corresponding l-value.
Note that \acrshort{p4} functions can combine \texttt{out} parameters and return
values.

Parameter directions are \acrshort{p4}'s way of introducing a limited form of
references, themselves a principled approach to pointers. Parametrized
\acrshort{p4} code follows a copy in / copy out calling convention, which makes
functions easy to reason about and limits side effecting behaviour of
\texttt{extern} methods. We will discuss this later in ?\todo{reference here}

.
\todo[inline]{subsubsubsection?}

Another construct for parametrized code are \emph{parser declarations}. A
\acrshort{p4} parser describes a \acrlong{fsm} whose job is to recognize valid
packets and load their headers into the storage facilities of the target. Parser
declarations define all the states of the \acrshort{fsm}, except the implicit
built-in \texttt{accept} and \texttt{reject} states further described in the
semantics section\todo{reference here}. Each parser has to contain at least the
initial \texttt{start} state. During parsing, the parser can manipulate local
state\todo{context?} in the form of variables and \texttt{extern}
instantiations. These are defined directly in the parser declaration block,
before listing any states, meaning that the type of this local context is
statically known and independent of the state the parser may appear in at
runtime. In other words, parser declarations follow lexical scoping.

States can introduce lexically-scoped local variables, but not additional
\texttt{extern}s. Other statements like conditionals, assignments, or method
calls are also allowed. The specification explicitly points out that specific
target architectures can place restrictions on the set of constructs and
operations a programmer can use within a parser.

\pfs parser declarations unrestricted by the target architecture may appear very
similar to regular imperative code. However, the important distinction between a
parser and other function-like constructs in the \pfs language are data
extraction, pattern-matching, error handling, and state transitions.

Data extraction moves information from the input packet into memory available
for pattern-matching further down the pipeline, typically registers or similar
containers. To facilitate this operation, the \acrshort{p4} core library
contains an intrinsic\todo{established term?} \texttt{extern}\todo{we should
definitely change this and the plural into a command} definition called
\texttt{packet\_in} which represents incoming packets. This special
\texttt{extern} cannot be manually instantiated\todo{make sure we explain
instantiation prior to this. Maybe externs should be described first in the
syntax section?}, but it is instantiated implicitly for every parser
parameter\todo{arg vs param} of type \texttt{packet\_in}. This allows parser
states to invoke the methods of this \texttt{extern}, shown in
Listing~\ref{lst:p4-prsr-packet-in}.

\begin{lstlisting}[
	caption={~The intrinsic \texttt{extern} that facilitates data extraction.},
	label=lst:p4-prsr-packet-in,
	captionpos=t,
	tabsize=4,
	float,
	abovecaptionskip=-\medskipamount,
	belowcaptionskip=\medskipamount,
	language=c
]
extern packet_in {
	void extract<T>(out T headerLvalue);
	void extract<T>(out T variableSizeHeader, in bit<32> varFieldSizeBits);
	T lookahead<T>();
	bit<32> length();  // This method may be unavailable in some architectures
	void advance(bit<32> bits);
}
\end{lstlisting}

\begin{lstlisting}[
	caption={~An example of data extraction in a \pfs parser.},
	label=lst:p4-prsr-extract,
	captionpos=t,
	tabsize=4,
	float,
	abovecaptionskip=-\medskipamount,
	belowcaptionskip=\medskipamount,
	language=c
]
struct Result { Ethernet_h ethernet;  /* more fields omitted */ }
parser P(packet_in b, out Result r) {
	state start {
		b.extract(r.ethernet);
	}
}
\end{lstlisting}

An example of extraction of fixed-width data can be seen in
Listing~\ref{lst:p4-prsr-extract}.

Data extraction and computation within parser states subsumes the stateful part
of a \acrlong{fsm}. Because isolated states would not be very useful on their
own, \acrshort{p4} parsers can specify state transitions using the
\texttt{transition} statement. The target of the transition can be either given
statically or computed. A missing \texttt{transition} statement at the end of a
state block implies a transition into the \texttt{reject} state.

Another parser-specific construct facilitates the computation of transition
targets by pattern-matching on extracted data. The \texttt{select} expression
tries to match a value against a number of patterns and evaluates to a state, if
successful. Otherwise, it triggers a runtime error with code
\texttt{error.NoMatch}. The patterns of a \texttt{select} expression can contain
integer literals, bit masks, ranges, don't-care values, tuples (when matching on
multiple values simultaneously), or, on some architectures, \textit{parser value
sets} specified at runtime by the control plane.\todo{one snippet to rule them
all, one snippet to find them, one snippet to bring them all and in the \LaTeX{}
bind them.}

While error-handling already happens implicitly in \texttt{select} expressions,
the \texttt{verify} statement, defined in Listing~\ref{lst:p4-prsr-verify},
allows the programmer to ergonomically check arbitrary assertions about the
parsed data. Passing \texttt{false} as the first argument to \texttt{verify}
immediately transitions to the \texttt{reject} state, setting the parser error
to the code given as the second argument. Otherwise, execution proceeds with the
next statement.

\begin{lstlisting}[
	caption={~The interface of the \texttt{verify} built-in.},
	label=lst:p4-prsr-verify,
	captionpos=t,
	tabsize=4,
	float,
	abovecaptionskip=-\medskipamount,
	belowcaptionskip=\medskipamount,
	language=c
]
extern void verify(in bool condition, in error err);
\end{lstlisting}

\begin{figure}\centering
	\begin{tikzpicture}[
		->,
		>=stealth,
		shorten >=1pt,
		auto,
		node distance=1cm,
		semithick
	]
		\tikzstyle{every state}=[fill=white,draw=black,text=black,minimum size=2em]
		\tikzstyle{cloud}=[draw=gray!20,circle,fill=gray!10,minimum height=2em]
		\tikzstyle{rejecting}=[state,double,minimum size=2em]

		\node[state] (start) {$\text{start}$};
		\node[state, below=of start] (q2) {};
		\node[state, left=of q2] (q1) {};
		\node[state, below=of q1] (q3) {};
		\node[state, below=of q2] (q4) {};
		\node[state, right=of q4] (q5) {};
		\node[state, accepting, below=1.5cm of q4] (accept) {$\text{accept}$};
		\node[rejecting, below=1.5cm of q3] (reject) {$\text{reject}$};

		\begin{pgfonlayer}{background}
			\node[cloud,fit=(q1)(q2)(q3)(q4)(q5)(start),inner sep=2pt] (cloud) {};
		\end{pgfonlayer}

		\path
			(start) edge          (q2)
			(q1)edge              (q2)
				edge              (q3)
			(q2)edge              (q4)
			(q3)edge              (q4)
				edge              (reject)
			(q4)edge              (q5)
				edge              (accept)
			(q5)edge              (accept)
			(q5)edge [bend right] (q3);
	\end{tikzpicture}
	\caption{An abstract overview of a \acrshort{p4} parser. The states inside
	the grey circle are accessible to user code.}
	\todo[inline]{add a loop}
	\label{fig:parser-overview}
\end{figure}


\begin{lstlisting}[
	caption={~A function declaration in \pfs.},
	label=lst:p4-fun,
	captionpos=t,
	tabsize=4,
	float,
	abovecaptionskip=-\medskipamount,
	belowcaptionskip=\medskipamount,
	language=c
]
bit<32> max(in bit<32> left, in bit<32> right) {
	return (left > right) ? left : right;
}
\end{lstlisting}

\todo[inline]{subparsers, header stacks}

\todo[inline]{we should really reorganise this to use bigger headings, this way
the individual constructs are lost in the text!}

While parsers execute at the very frontier of the pipeline, the bulk of packet
processing happens in match-action units\todo{previously mentioned?}.
\acrshort{p4} has another parametrized construct for expressing entire sequences
of pattern-matching and rewriting steps: \texttt{control} blocks\todo{Add an
example control block listing}.

A control block has a name and can take type and value parameters. The body of a
control block begins with declarations of constants, variables, and
instantiations. It follows with definitions of \emph{\texttt{action}s}.

A \acrshort{p4} action\todo{example listing} is a piece of code that directly
manipulates packet headers. At a first glance, actions should be immediately
familiar to imperative programmers, since they resemble functions with no return
value. Their bodies are each comprised of a series of statements. These are
executed sequentially\footnote{At least from the perspective of the
programmer.}. There are two important restrictions on action parameter lists.

\begin{enumerate}
	\item Parameters with no direction must all come at the end of the parameter
	list. These directionless parameters indicate \emph{action
	data}\todo{explain what that is}.
	\item Action parameters cannot have \texttt{extern} types.
\end{enumerate}

The second major \acrshort{p4} construct that appears exclusively\todo{right?}
within control blocks are \emph{tables}. Whereas an action specifies the
operations performed on packet headers when pattern-matching succeeds, a table
informs a match-action unit of the target how to perform the pattern-matching
and what actions to invoke.

Functional programmers will note that tables and actions together seem like a
low-level decomposition of pattern-matching from Haskell or Scala. This is a
good analogy, except that, whereas pattern-matching constructs in a programming
language are typically fixed at compilation time, \acrshort{p4} tables are
configurable at runtime by the control plane. Furthermore, \acrshort{p4}'s
pattern-matching operates at the bit level and offers certain string-like
facilities.\todo{this whole thing should be like an aside or something}

A \texttt{table} declaration is simultaneously an instantiation in the enclosing
control block. The declaration lists various properties of the table, given as
key-value pairs. It has at least the mandatory \texttt{key} and \texttt{actions}
properties, which specify an expression used for computing the lookup key and
the set of actions the table can invoke, respectively. A table declaration can
optionally also include \texttt{default\_action}, \texttt{entries}, and/or
\texttt{size} properties, which specify the action invoked when no other actions
match\footnote{The \texttt{default\_action} property defaults to the built-in
\texttt{NoAction} when not specified.}, a predefined set of entries for the
table, and the desired size for the table. Compilers can choose to extend this
standard set of properties with additional target-specific key-value pairs.

The programmer can optionally declare table properties as \texttt{const}. This
keyword ensures that the control plane cannot change the property's value at
runtime. Somewhat confusingly, some properties are implicitly \texttt{const},
including the standard properties \texttt{key}, \texttt{actions}, and
\texttt{size}. On these, the \texttt{const} keyword has no effect.

\todo[inline]{some sort of heading for keys here}

\begin{description}

\item[\texttt{key}] The \texttt{key} table property specifies the scrutinee of
pattern-matching. Grammatically, the key is a sequence of rows where each row
contains the expression to match on, the kind of matching to perform, and a list
of optional annotations.

The possible kinds of pattern-matching are \texttt{exact}, \texttt{ternary}, and
\texttt{lpm}, all defined as part of the \texttt{match\_kind} construct in the
core library\todo{do we discuss that prior to this point?}. A
\texttt{match\_kind} behaves like an enumeration, it lists a number of
mutually-exclusive variants. The semantics of the \texttt{match\_kind} variants
is not given, it depends on the compiler and target architecture.

Key elements can be renamed with optional \texttt{@name} annotations to give
them readable names in the control plane \acrshort{api}.

\item[\texttt{actions}] The \texttt{actions} table property lists all the
actions a table can invoke at runtime, separated by semicolons.

The specification mandates that action names of actions in the \texttt{actions}
list have to be distinct. That is, two actions of the same name cannot appear in
the \texttt{actions} list, even if they come from different scopes and could be
disambiguated with fully qualified paths in different contexts.

An overview of the valid and illegal usages of the \texttt{actions} table
property is given in Listing~\ref{lst:p4-table-actions}. The examples highlight
the distinction of action parameters with and without direction. The
directionless parameters, also known as \emph{action data}, are used to
communicate data between the control plane and the data plane at runtime, and
therefore cannot be bound in the \acrshort{p4} program. Haskell enthusiasts may
note that the distinction between compile time and runtime -bound values is
somewhat reminiscent of second-class functions.

\begin{lstlisting}[
	caption={~Use of the \texttt{actions} table property in \pfs.},
	label=lst:p4-table-actions,
	captionpos=t,
	tabsize=4,
	float,
	abovecaptionskip=-\medskipamount,
	belowcaptionskip=\medskipamount,
	language=c
]
action a(in bit<32> x) { /* body omitted */ }
bit<32> z;
action b(inout bit<32> x, bit<8> data) { /* body omitted */ }
table t {
	actions = {
		// a;
		//  - illegal, x parameter must be bound
		a(5);  // binding a's parameter x to 5
		b(z);  // binding b's parameter x to z
		// b(z, 3);
		//      - illegal, cannot bind directionless data parameter
		// b();
		//  -- illegal, x parameter must be bound
		// a(table2.apply().hit ? 5 : 3);
		//   -------------- illegal, cannot apply a table here
	}
}
\end{lstlisting}

\item[\texttt{default\_action}] The \texttt{default\_action} table property
specifies the action invoked when no other actions match. The value of this
property is an expression that invokes the action in question, binding
parameters with directions similarly to an \texttt{actions} list entry. It
introduces an interesting redundancy in the \pfs language, since the programmer
needs to also include the default action in the \texttt{actions} list. Moreover,
the default action and the entry in the list have to be syntactically identical,
except for the directionless parameters. Since there is no action data for the
default action at runtime, the \texttt{default\_action} property has to specify
them all. The directionless parameters are evaluated at compile time.

A table without a \texttt{default\_action} property that does not match a given
packet has no impact on packet processing, it is effectively skipped.

\item[\texttt{entries}] The optional \texttt{entries} property preconfigures a
table's entries. This can serve to either initialize the table or, with the help
of \texttt{const}, turn it into a static construct that cannot be modified on
the control plane side.

\item[\texttt{size}] The optional \texttt{size} property indicates the number of
table entries the table should support at runtime. It is a compile-time known
integer. The interpretation of \texttt{size} depends largely on details of the
target architecture, casting some doubt on why it was included in the core
language. Some targets may require every table to specify a size, others may use
it merely as a hint during dynamic resource allocation, others still may
guarantee that a successfully compiled table can fit at least \texttt{size}
entries. In general, none of these are the case.

\end{description}

Tables can specify additional properties outside of the base set provided by the
\pfs specification\footnote{The specification lists several motivated
examples.}.

\subsection{Semantics}

The semantics of \pfs is defined entirely in terms of abstract machines
executing imperative code. A conforming compiler is free to rewrite the \pfs
program as long as it maintains the observable behaviour of all abstract
machines involved. Unfortunately, the specification gives no formal treatment
of these machines; they are described only in natural language and pseudocode.

\todo[inline]{point out how the language tries to avoid undefined behaviour
and makes arithmetic typing rules saner and safer with no undef. behaviour}


\subsubsection*{Parser abstract machine}

A parser conceptually manipulates a \texttt{ParserModel} data structure.

\begin{lstlisting}[
	caption={~The conceptual model of the state of a \acrshort{p4} parser.},
	label=lst:p4-parser-model,
	captionpos=t,
	tabsize=4,
	float,
	abovecaptionskip=-\medskipamount,
	belowcaptionskip=\medskipamount,
	language=c
]
ParserModel {
	error       parseError;
	onPacketArrival(packet p) {
		ParserModel.parseError = error.NoError;
		goto start;
	}
}
\end{lstlisting}

The meaning of \texttt{accept} and \texttt{reject} states is
architecture-dependent. For example, a rejected packet may be dropped or
passed to the next block of the processing pipeline.

Parser transitions are analogous to \texttt{goto} statements or parameter-less
tail calls.

\subsubsection*{Calling convention}

In order to simplify reasoning about function-like constructs, \pfs follows the
\emph{call by copy in / copy out} calling convention. The specification
describes the exact steps a conforming implementation needs to go follow in
general\footnote{Compiler optimizations could, in specific cases, eliminate some
of these steps, provided prior analysis determines such optimizations correct.}
to evaluate a function call expression.

\begin{displayquote}
	\emph{\begin{enumerate}
		\item Arguments are evaluated from left to right as they appear in the
		function call expression.
		\item If a parameter has a default value and no corresponding argument
		is supplied, the default value is used as an argument.
		\item For each \texttt{out} and \texttt{inout} argument the
		corresponding l-value is saved
		(so it cannot be changed by the evaluation of the following arguments).
		This is important if the argument contains indexing operations into a
		header stack.
		\item The value of each argument is saved into a temporary.
		\item The function is invoked with the temporaries as arguments. We are
		guaranteed that the temporaries that are passed as arguments are never
		aliased to each other, so this ``generated'' function call can be
		implemented using call-by-reference if supported by the architecture.
		\item On function return, the temporaries that correspond to
		\texttt{out} or
		\texttt{inout} arguments are copied in order from left to right into the l-values
		saved in Step 3.
	\end{enumerate}}

	-- \citetitle{p416:v123:spec} \cite{p416:v123:spec}
\end{displayquote}

The calling convention gives function calls\todo{I really need some sort of
universal definition of function-like constructs in p4 that I stick to
throughout the text} a semantics that ensures arguments cannot alias one another
and impure functions cannot hold references to \acrshort{p4} code. This design
is ultimately motivated by the need to control the side effects of
\texttt{extern} functions and methods, which are arbitrarily powerful. The copy
in / copy out calling convention lets compilers reason about programs in the
presence of \texttt{extern}s.

\subsubsection*{Control blocks}

Invoking actions: actions can be invoked either implicitly or explicitly.
Implicit invocations come from tables during match-action processing. Explicit
invocations are calls from \texttt{control} blocks or other \texttt{action}s. In
the explicit invocation case, directionless parameters follow \texttt{in}
parameter semantics.

A table can be invoked explicitly by calling its \texttt{apply} method. This
method is synthetic; it is generated by the compiler. Its return type is a
synthetic \texttt{struct} shared with other actions in a given table. The
compiler generates an \texttt{enum} and a \texttt{struct} for each table, both
of which can be seen in Listing~\ref{lst:p4-table-synthetics}.

\begin{lstlisting}[
	caption={~The synthetic \acrshort{p4} code generated for a table.},
	label=lst:p4-table-synthetics,
	captionpos=t,
	tabsize=4,
	float,
	abovecaptionskip=-\medskipamount,
	belowcaptionskip=\medskipamount,
	language=p4
]
enum action_list(T) {
	// the name of each enum variant is the name of an action
	action_1,
	action_2,
	...
	action_n
}
struct apply_result(T) {
	bool hit;
	bool miss;
	action_list(T) action_run;
}
\end{lstlisting}

Here we see a rationale for the requirement that an action list cannot contain
two actions with the same name. The restriction ensures that the corresponding
enum variants never clash.

The fields in the \texttt{apply\_result(T)} \texttt{struct} indicate whether the
table hit or missed and which action was executed\footnote{\texttt{action\_run}
will be set to the variant corresponding to the table's \texttt{default\_action}
on a table miss.}. The \texttt{hit} and \texttt{miss} fields are
mutually-exclusive, exactly one of them is set when the action returns.

\subsubsection*{The match-action pipeline abstract machine}

Although the \pfs specification does not define an abstract machine for the
match-action pipeline, it does describe the semantics by analogy to imperative
programs.

The body of a control block is executed serially, as a sequence of imperative
statements. The syntax restricts the control block body such that the bulk of
code within is limited to a special sub-block introduced with the \texttt{apply}
keyword. The \texttt{apply} sub-block is the only place where tables can be
explicitly invoked. The control can also invoke other controls or explicitly
invoke actions by calling their \texttt{apply} methods, although it cannot
invoke parsers.

The \texttt{apply} block can contain \texttt{return} and \texttt{exit}
statements. A \texttt{return} statement immediately terminates execution of the
control block and returns control to the caller. An \texttt{exit} statement
terminates execution of the entire control block call chain.

A control block can invoke subroutines in the form of other controls during its
execution. This requires prior instantiation of the sub-controls, either in the
calling control block's body or prior to invoking a calling control block that
is passed a control instance as a parameter. However, \pfs offers syntax
sugar\todo{move to the syntax section if we decide to keep it} for the common
case of using a sub-control exactly once. This shortcut is known as \emph{direct
type invocation} and works even for controls with constructor
parameters\todo{mention those prior to this point}. An example can be seen in
Listing~\ref{lst:p4-direct-type-invocation}.

\begin{lstlisting}[
	caption={~An example of direct type invocation.},
	label=lst:p4-direct-type-invocation,
	captionpos=t,
	tabsize=4,
	float,
	abovecaptionskip=-\medskipamount,
	belowcaptionskip=\medskipamount,
	language=p4
]
control Callee(/* parameters omitted */) { /* body omitted */ }

control Caller(/* parameters omitted */) {
	apply {
		// Callee can be treated as an instance
		Callee.apply(/* arguments omitted */);
	}
}

// the Caller definition above desugars to the following

control Caller(/* parameters omitted */) {
	// local instance of Callee
	@name("Callee") Callee() Callee_inst;
	apply {
		// Callee_inst is simply applied
		Callee_inst.apply(/* arguments omitted */);
	}
}

\end{lstlisting}

\subsubsection*{Constructor parametrization}

Although parsers and control blocks take parameters, these only define the
interface between \acrshort{p4} code and the target architecture. This makes
abstraction difficult, as expressing control blocks or parsers with similar
structure requires either code duplication or a reliance on preprocessor
macros\footnote{Which would be a rather dreadful affair.}. To address this
problem, \pfs supports \emph{constructor parameter lists}. Constructor
parameters are given in secondary, optional parameter lists, available to
control blocks and parsers. Syntactically, they follow the regular parameters of
the given construct. However, at the site of instantiation, the values of
constructor parameters directly follow the type name.

All constructor parameters must be directionless. Instantiations must bind all
constructor parameters to compile-time known values. Constructor parametrization
offers a light form of templating that is sufficient for many use cases and
saves the user from substituting generic parsers and control blocks by hand.

\begin{lstlisting}[
	caption={~An example of a parser with constructor parameters.},
	label=lst:p4-control-params,
	captionpos=t,
	tabsize=4,
	float,
	abovecaptionskip=-\medskipamount,
	belowcaptionskip=\medskipamount,
	language=p4
]
// a parser with one constructor parameter
parser GenericParser(packet_in b, out Packet_header p)
                    (bool udpSupport) {
	// body omitted
}
...
// the instantiation specifies all constructor parameters
// topParser is a GenericParser where udpSupport = false
GenericParser(false) topParser;
\end{lstlisting}


\subsubsection*{Deparsing}

Deparsing is the inverse of parsing, i.e. the process of converting packet
headers from their semantic representation back into a sequence of bytes. A
careful reader will observe that the original \acrshort{p4}\textsubscript{14}
forwarding model from Figure~\ref{fig:abstract-forwarding-model} does not
include this step. Indeed, a deparser is not necessary if the device does not
rewrite packets. For example, certain architectures may rely solely on metadata
to communicate information about packet forwarding from the data plane and
possibly include a rewriting step outside the \acrshort{p4} model.

Nevertheless, deparsing is a common step for many network processors and \pfs
fully supports it. However, the language does not include any special syntax for
deparsing, which speaks somewhat to its generality. \pfs deparsers are expressed
by means of control blocks combined with the special \texttt{packet\_out}
\texttt{extern}. A value of type \texttt{packet\_out} denotes a buffer
containing the packet to be sent out.

Listing~\ref{lst:p4-deparser} shows an example deparser. The \texttt{emit}
method of the \texttt{packet\_out} \texttt{extern} appends the given data to the
packet buffer. The behaviour of a call to \texttt{emit} has different semantics,
depending on the type of data being emitted.

\begin{description}
	\item[Headers] are emitted only if valid\todo{discuss header validity in the
	parser semantics section}. If the header is invalid, the call to
	\texttt{emit} behaves like a no-op.

	\item[Header stacks] have their elements emitted recursively.

	\item[Header unions and \texttt{struct}s] have their fields emitted
	recursively.

	\item[\texttt{enum}s and \texttt{error}s] cannot be serialized. It is
	illegal to invoke \texttt{enum} directly or indirectly (through implicit
	recursive behaviour) on these types.

	\item[Base types] are emitted as-is, but only indirectly. It is illegal to
	explicitly invoke \texttt{emit} on a base type.
\end{description}

The recursive behaviour of \texttt{emit} always follows the order of
declarations in the \acrshort{p4} source code.

\begin{lstlisting}[
	caption={~An example of a \pfs deparser.},
	label=lst:p4-deparser,
	captionpos=t,
	tabsize=4,
	float,
	abovecaptionskip=-\medskipamount,
	belowcaptionskip=\medskipamount,
	language=p4
]
control Deparser(inout headers hdr, packet_out packet) {
	apply {
		packet.emit(hdr.ethernet);
		packet.emit(hdr.ipv4);
		if (hdr.ipv4.isValid()) {
			if (hdr.ipv4.protocol == IPProtocol.UDP) {
				packet.emit(hdr.udp);
			}
			if (hdr.ipv4.protocol == IPProtocol.TCP) {
				packet.emit(hdr.tcp);
			}
		}
	}
}
\end{lstlisting}

\begin{lstlisting}[
	caption={~The \texttt{packet\_out} \texttt{extern}.},
	label=lst:p4-packet-out-extern,
	captionpos=t,
	tabsize=4,
	float,
	abovecaptionskip=-\medskipamount,
	belowcaptionskip=\medskipamount,
	language=p4
]
extern packet_out {
	void emit<T>(in T data);
}
\end{lstlisting}


\subsubsection*{The \acrshort{p4} abstract machine}

Throughout this section, we have occasionally referred to some values as
``evaluated at compile time'' or ``compile time -bound.''\todo{verify these
references actually happened} A reader familiar with staged programming,
dependent types, or C++ template instantiation may worry for the compiler
performance that unrestricted compile-time evaluation entails. Fortunately, the
\pfs specification has a rigorous definition for these terms and clarifies the
extent to which a \acrshort{p4} compiler must evaluate a program during
translation.

\emph{Compile-time known values} are generally constants and stateless
constructs that themselves only depend on other compile-time known values. Since
\acrshort{p4} has no loops, no recursion, and no complex metaprogramming
facilities, the set of compile-time known values is finite and can be evaluated
in a single pass.\todo{we don't need to explain \emph{everything}, right?}

A \acrshort{p4} program is evaluated in two stages, statically at compile time
and dynamically at runtime. The notion of compile-time evaluation in
\acrshort{p4} is closely related to resource allocation of the target device.
Static evaluation proceeds in the order that declarations appear in the source
file. It begins at the top level and recurses into lexical scopes.

\todo[inline]{finish this}

\subsubsection*{Control plane \acrshort{api}}

A \acrshort{p4} compiler generates methods for interacting with data plane
constructs that can be in any way controlled or configured at runtime. These are:

\begin{itemize}
	\item value sets
	\item tables
	\item keys
	\item actions
	\item \texttt{extern} instances
\end{itemize}

To disambiguate references from the control plane to these entities, \pfs
requires that each such entity has a unique fully qualified name. Additionally,
control block and parser instances also need unique names, because they contain
constructs from the above list.

The \pfs specification lays out in detail what names are given to various syntax
forms, but these technicalities are not relevant for the purposes of this
thesis. Generally, the control plane name corresponds to the accessor syntax of
the construct in \acrshort{p4} code. For example, a slice of the four least
significant bits of the bit string \texttt{s} (\texttt{s[3:0]}) is named
\texttt{s[3:0]} in the control plane \acrshort{api}. The only thing to note here
is that in cases where a name is not given straightforwardly, the compiler
requires a \texttt{@name} annotation to be attached to the construct.
\texttt{@name} can also be used to rename other constructs. The second
annotation controlling naming is \texttt{@hidden}, which can be used to hide
constructs from the control plane \acrshort{api}. Hidden constructs are not
subject to the unique name requirement.

\subsubsection*{Dynamic evaluation}

A \acrshort{p4} program defines the parsers, control blocks, \texttt{extern}s,
and other constructs that make up the data plane. However, it is up to the
target architecture to decide how and when are all these execution blocks
evaluated at runtime. Packet processing systems often operate concurrently,
so the \pfs specification lays out\todo{finish}

%--------------------------------------%
\chapter{Language Server Architecture}
%--------------------------------------%

\begin{chapterabstract}
	\dots in which we introduce language servers in general and the
	\acrlong{lsp} in particular, outline their relationship to other classes of
	language tooling, and look at high-level architectural decisions that their
	use-cases imply.
\end{chapterabstract}

Language servers have a lot in common with compilers. Like compilers, they have
to build up a semantic model of a program and provide useful diagnostics for
invalid or suspicious input along the way. Unlike a compiler, a language server
needs to maintain the semantic model over the course of an editing session.
Rather than operating in batch mode, a language server runs continuously and is
expected to provide feedback within milliseconds.

However, language servers need neither to produce compilation artifacts nor to
optimise the programs they process. Instead, they are effectively
special-purpose query engines. Their output is a data structure optimised for
fast querying, such as finding references, definitions, or providing
context-sensitive completion suggestions. From that point of view, it could seem
like a language server is merely a compiler frontend with very little backend
logic. Unfortunately, this is not the case.

The requirement for real-time feedback to the developer is the primary
constraint on a language server's design. For all but the most basic languages
and features, instant feedback requires an incremental computation approach and
management of state that persists across updates to source files. The second
most important consideration, and one to a large extent not shared with
compilers, is resilience to errors. While a compiler generally expects
well-formed input, a language server deals with all sorts of intermediate states
of a document, including files with many syntactic errors, invalid encodings, or
unsaved buffers outside the filesystem.

To make matters worse, while compiler frontend implementations are often guided
by a language specification\footnote{Even if, usually, an informal one.}, the
space of invalid programs is unconstrained. Developers of language servers have
to guess what intermediate states a program goes through during development and
how to respond to them. Generating meaningful semantic models from invalid input
is a challenging task, but doing so is often crucial for developers. For
example, smart auto-completion in statically typed languages is expected to
provide type-correct suggestions, even as the document being edited is
type-incorrect, and often semantically or even syntactically invalid.

In this chapter, we explore the status quo of language servers, how they differ
from conventional batch-processing compilers, how this gap may narrow in the
future, and what makes building interactive language tooling difficult.

%----------------------------------------%
\section{The fruits of semantic support}
%----------------------------------------%

The largest language servers conforming to \acrshort{lsp} offer a wide variety
of features. The vast majority of the \acrshort{api} surface is optional,
however. Upon establishing a connection, the client and server exchange
information about their respective capabilities, establishing a subset of
\acrshort{lsp} they both support.

The functionality of \acrshort{lsp} comes in two main flavours: code
comprehension and coding features. The former subsumes utilities which ease
reading and navigating through code, such as Hover (where the editor displays
details about an object under the pointer), Go to Definition, Find References,
etc. Utilities like diagnostics, auto-completion, or code actions are more
relevant to the programmer at the time they are authoring code and belong in the
latter category.

Next, we will delve into both categories and take a closer look at what they
offer.


\subsection{Code comprehension in \pdfacrshort{lsp}}

\newcommand{\org}{\pdftooltip{(original)}{This request was present in the
original LSP specification.}}

Code comprehension functionality takes up the majority of \acrshort{lsp}'s
\acrshort{api} surface and has been growing during the protocol's evolution. At
the time of writing, the latest stable \acrshort{lsp} release is version 3.17.
The central features, already present in the initial specification of the
protocol, are Go to Definition, Find References, Document Highlight, Document
Link, Hover, Code Lens, and Document Symbols.

\begin{figure}[h]
	\includegraphics[width=1.00\textwidth]{resources/code_find_references.png}
	\caption{Find References in Visual Studio Code via rust-analyzer.}
\end{figure}

Go to Definition and Find References are
present in some of the oldest code comprehension tools, dating back at least to
the Unix utility \texttt{ctags}\cite{exuberant-ctags}. These let the editor jump
from a symbol's use-site to its definition and vice versa, just as their names
imply. \acrshort{lsp}'s Document Highlight request does not provide syntax
highlighting, rather, it serves to visually assist the programmer with locating
references of a given symbol without having to explicitly invoke the Find
References feature. Document Highlight could be merged with Find References
functionality, but the \acrshort{lsp} maintainers choose to keep them separate
and allow Document Highlight to report imprecise (``fuzzy'') matches. A response
to the Document Link request lists the hyperlinks embedded in the document.
Document Symbols provides a potentially hierarchical overview of the symbols of
a document, which serves the ``outline'' feature of modern editors: a tree
overview of a program's constructs, such as modules, classes, fields, and
methods. The Hover feature provides additional contextual information when
navigating code. It is typically implemented by the client rendering a pop-up
box of documentation for a given symbol. Finally, Code Lens is a versatile
editor feature for displaying additional information at a given position in a
document, such as the number of references of a type or the code metrics of a
procedure\cite{codelens-comparison}. It can trigger an action when activated,
which is used by some servers to run tests or open a Find References dialog.

\begin{figure}[t]\centering
	\begin{multicols}{2}
	\begin{enumerate}
		\item Go to Declaration
		\item Go to Definition \org
		\item Go to Type Definition
		\item Go to Implementation
		\item Find References \org
		\item Prepare Call Hierarchy
		\item Call Hierarchy Incoming Calls
		\item Call Hierarchy Outgoing Calls
		\item Prepare Type Hierarchy
		\item Type Hierarchy Super Types
		\item Type Hierarchy Sub Types
		\item Document Highlight \org
		\item Document Link \org
		\item Document Link Resolve \org
		\item Hover \org
		\item Code Lens \org
		\item Code Lens Refresh
		\item Folding Range
		\item Selection Range
		\item Document Symbols \org
		\item Semantic Tokens
		\item Inlay Hint
		\item Inlay Hint Resolve
		\item Inlay Hint Refresh
		\item Document Color
	\end{enumerate}
	\end{multicols}

	\caption{Code comprehension -related requests in \acrshort{lsp} 3.17.}
	\label{fig:comprehension-requests}
\end{figure}


\subsection{Coding features in \pdfacrshort{lsp}}

A shorter but no less important range of \acrshort{api} calls supports the
developer right when they are authoring code. The main features are
auto-completion, signature help, formatting, and symbol renaming.

\begin{figure}
	\includegraphics[width=1.00\textwidth]{resources/code_signature_help.png}
	\caption{Signature help in VS Code for Rust shows a pop-up with
	documentation as well as the signature of the callee, highlighting the
	parameter under cursor.}
\end{figure}

Auto-completion offers to fill in code as the programmer is typing, supports
ranking results based on their relevance, on both the server and the client
side, and can include a ``quick info'' description for each option. Signature
help shows parameter name and type information when calling a procedure, method,
or function. Formatting allows a language server to rewrite a document upon
request, for example to conform to a particular code style. The \acrshort{lsp}
formatting functionality can format either the entire document, a selected
range, or reactively an arbitrary part of the document as the user types.
Finally, symbol renaming performs a context-sensitive workspace-wide rename of a
given symbol.

\begin{figure}[t]\centering
	\begin{multicols}{3}
	\begin{enumerate}
		\item Inline Value
		\item Inline Value Refresh
		\item Moniker
		\item Completion Proposals \org
		\item Completion Item Resolve \org
		\item Publish Diagnostics
		\item Pull Diagnostics
		\item Signature Help \org
		\item Code Action
		\item Code Action Resolve
		\item Color Presentation
		\item Formatting \org
		\item Range Formatting \org
		\item On type Formatting \org
		\item Rename \org
		\item Prepare Rename
		\item Linked Editing Range
	\end{enumerate}
	\end{multicols}

	\caption{Coding language features in \acrshort{lsp} 3.17.}
	\label{fig:coding-requests}
\end{figure}

Later \acrshort{lsp} revisions added high-level features not universally
applicable to all programming languages. For instance, version 3.16 added
\emph{linked editing}, which some conforming implementations use to update
opening and closing \acrshort{xml} tags seamlessly without the user specifically
triggering a rename action. Version 3.17 introduced \emph{type hierarchy}
requests, relevant only to programming languages with subtyping. On the other
hand, more general facets of the protocol see creative use in unintended
contexts. One example is the use of lenses and special comments in the Haskell
Language Server project\cite{haskell-ls} to provide \acrshort{repl}-like
functionality. Another is the LT$_\text{E}$X Visual Studio Code
extension\cite{vscode-spellcheck}, which provides spell and grammar checking in
Markdown and \LaTeX{} documents, as well as in programming language comments.
Even though \acrshort{lsp} has no built-in support for extracting the comments
of a document or for spell checking in general, LT$_\text{E}$X achieves this
with a combination of non-\acrshort{lsp} \acrshort{api}s and by leveraging
\acrshort{lsp} diagnostics and code actions to provide suggested spellings.

\begin{figure}[h]\centering
	\includegraphics[height=0.3\textheight]{resources/code_haskell_repl.png}
	\caption{The Haskell Language Server project supports in-editor expression
	evaluation in comments prefixed with \texttt{>>>} and checking QuickCheck
	properties in comments starting with \texttt{prop>}.}
\end{figure}


\begin{tcolorbox}[
	title={\textbf{Language servers for popular technologies}},
	colback=decoration!5!white,
	colframe=decoration,
	fonttitle=\bfseries,
	arc=0pt,
	outer arc=0pt,
	boxrule=0.5pt,
	top=2pt,
	bottom=2pt,
	left=2pt,
	right=2pt,
	enlarge top by=1.5\baselineskip,
	enlarge bottom by=1.5\baselineskip
]

Many of the most widely used programming languages have corresponding language
server implementations. The table below presents VS Code extensions and
programming languages in order of their install count and popularity,
respectively. The number of extension installations is tracked by the VS Code
marketplace
(\url{https://marketplace.visualstudio.com/search?target=VSCode&category=Programming%20Languages&sortBy=Installs}).

\vspace{1em}

\begin{center}
\begin{tabular}{|r|c|c|}
	Rank & VS Code extension & Popular languages, according to StackOverflow\cite{stackoverflow-survey-2022}
	\tabularnewline \hline
	1  & Python     & Python     \tabularnewline
	2  & C/C++      & Java       \tabularnewline
	3  & Java       & C\#        \tabularnewline
	4  & C\#        & C/C++      \tabularnewline
	5  & Go         & PHP        \tabularnewline
	6  & PHP        & PowerShell \tabularnewline
	7  & PowerShell & Go         \tabularnewline
	8  & Dart       & Rust       \tabularnewline
	9  & Ruby       & Dart       \tabularnewline
	10 & Rust       & Ruby       \tabularnewline
\end{tabular}
\end{center}

\vspace{1em}

We have chosen to exclude some technologies from this comparison. JavaScript,
TypeScript, CSS, and HTML support is built into VS Code and therefore does not
show up in VS Code marketplace statistics for installed language extensions.
SQL, while popular, has too many dialects to present a faithful picture through
the lens of extension installations alone. These technologies were in turn
filtered out from StackOverflow's list of most popular languages to highlight
the differences in ranking.
\end{tcolorbox}

%----------------------------------------%
\section{Lessons from the compiler world}
%----------------------------------------%

We have previously established how closely does the task of implementing
language servers relate to writing compilers. Let us expand on the possible
architectural similarities of the two kinds of language tools in this section.

\subsection{The pipeline}

Traditional compiler architectures build around a pipeline approach. The
compiler begins with a frontend, typically composed of a lexer\footnote{Although
the lexer/parser distinction is maintained, partly for historical reasons, in
many modern compilers, the jobs of lexers and parsers are conceptually
identical. These compositional algorithms transform input of one type (usually a
string) into output of another, verifying certain properties along the way.
Concretely, they match the input against some
grammar\footnotemark.}\footnotetext{A keen reader will note that without
restrictions on the \emph{class} of grammars, this statement makes the
description no more concrete.}, a parser, and a step of semantic analysis. The
second major part is the backend, a combination of an optimiser and a code
generator. The lexer ingests bytes of text, turning them into tokens for the
parser. The parser matches the tokens against the productions of a given
language's grammar, producing \acrlong{ast}s. Semantic analysis then verifies
that the parsed program adheres to the language's semantic restrictions. For
example, semantic analysis ensures that variables can only be referenced after
their declaration or definition, \texttt{break} statements can only appear in
the bodies of loops, and the program follows typing rules.

Is it the frontend's responsibility to identify and report user errors,
terminating the compiler pipeline as soon as it identifies invalid input. This
is a practical choice, since later stages of the compiler can assume the program
valid and not worry about possible errors. Furthermore, computation of the
backend stages would be wasted on invalid input anyway, as the compiler could
give no guarantees about the produced executable form\footnote{Whether that is
machine code in a platform-specific executable, bytecode for a virtual machine,
source code in the case of transpilers, or something else entirely.}.

One important step that typically happens on the frontend/backend boundary is
\emph{lowering}. This process turns the \acrshort{ast} obtained and validated by
previous stages into the compiler's \emph{\acrfull{ir}}. The \acrshort{ir}
represents a simplified language devoid of syntactical sugar and various
programmer-facing niceties. For example, various types of conditional
expressions are usually represented by a single \acrshort{ir}
instruction\footnote{Which may in fact conceptually be an instruction, or a node
of the \acrshort{ir} graph.}. Similarly, different forms of loops lower to a few
canonical translations. Freedoms in the source-level code style are eliminated
to make reasoning about the program further down the pipeline easier. Nested
expressions are flattened using short-lived local variables, often imposing an
evaluation order. Overall, \acrlong{ir}s tend to be semantically closer to the
target architecture.

After lowering, the backend takes over. The optimizer, if it is involved in the
compilation, iteratively transforms the \acrshort{ir} in a series of analysis
and rewriting steps that attempt to minimise certain cost functions\footnote{The
actual cost functions may vary based on what steps the optimizer takes. For
example, the metrics used for inlining can differ from heuristics consulted for
loop-invariant code motion.}. Optimization can sometimes cross the boundaries of
source-level code. That is, certain \acrshort{ir} programs do not have a
corresponding source program, because the lowering phase is not a surjective
mapping. This is a necessary freedom for eliminating overhead in implementations
of high-level programming languages, but makes mapping issues in the final
executable more difficult\footnote{Which is of major concern for debugging, and
consequently one of the primary reasons why native executable debuggers tend to
operate better on programs compiled with fewer optimizations.}.

Finally, the pipeline terminates in a code generation phase which emits the
final executable form. This step requires rewriting the \acrshort{ir} program in
order to fit the target constraints. The amount of work the compiler needs to do
in this last stage depends on the semantic differences between the structure of
the \acrlong{ir} and the target language. It could be as simple as writing the
\acrshort{ir} to an output file, if the intermediate and target languages
perfectly match. For conventional processor targets, however, code generation
necessitates at least instruction selection and register allocation, as well as
maintaining some level of conformance to standard calling conventions.

\subsubsection*{Where the pipeline falls short}

In recent years, the feedback loop from writing code to executing it has gotten
shorter and shorter. The case for early feedback is simple: programmers want
results as soon as possible. Moreover, since the programmer typically uses the
compiler in an online fashion, changes to the source files tend to be small and
local. Compiling small changes in the source program often only requires small
changes to the output. With proper caching, most information can be reused from
previous runs of the compiler.

With semantic language support receiving more and more attention, the overlap
between compilers and interactive language tooling is only getting clearer. Both
classes of programs now need to maintain and incrementally update databases of
semantic information about the codebase in order to respond quickly to user
requests. With semantic information at hand, it is only natural for both classes
to integrate tightly with editors and development environments to provide
features reliant on high-level information. Both compilers and language servers
should be resilient to errors in the user input and continue processing as far
as is practical, for example to report semantic errors even in the presence of
syntactical issues. Just as with optimisers and other components useful across
many frontends, reimplementing a lot of complicated functionality is tiresome
and unnecessary.

Unfortunately, the well-established pipeline approach to compiler construction
is, despite its many innovations, difficult to adapt to the modern incremental
workloads. The interfaces between stages do not share a principled, universal
structure, requiring each phase to implement its own variant of caching and
cache-invalidation. Running phases concurrently for independent sections of the
input program can be problematic, because older pipeline-based compilers often
rely on global variables\footnote{This is no fault of the architecture as a
whole, but it is an important practical consideration.}.

Conventional incremental compilers choose a granularity of input file,
compilation unit, or module, and often run several pipelines in parallel,
culminating in a sequential step that combines the constituent products into the
final executable. This approach achieves very short compilation and
recompilation times for many programs, but tends to produce suboptimal code,
since the optimiser can only see a subset of the code and is limited in
attempting whole-program optimisation and cross-module inlining. Nowadays,
linkers (the usual last step in the production of executables) counter this
downside with \acrfull{lto}, trading linking time for better quality code.
Running many pipelines in parallel also tends to increase the size of the
intermediate compilation artifacts, because dead code elimination cannot safely
remove unused definitions, possibly referenced by other compilation units.

\subsection{The pipeline as a sequence of queries} \label{subsec:query-pipeline}

With these new developments and performance constraints in mind, recent compiler
construction techniques put incremental computation at the centre of their
design. For example, Roslyn\footnote{This product is officially called \emph{The
.NET Compiler Platform SDK} but it is arguably better known under the Roslyn
codename.}, a collection of compilers and analysers for C\# and Visual Basic,
builds heavily on incremental computation. A Roslyn compiler takes centre stage
in a number of interactive and batch applications, maintaining a semantic model
of the codebase. Higher-level tools then query and update this model via several
\acrshort{api}s designed for static analysis, code refactoring, and other
use-cases\cite{roslyn-apis}. The entire stack is optimised for interactive use.
For example, the parsers utilise a combination of persistent ``green''
\acrlong{ast}s and transient ``red'' trees. The second, ephemeral type is built
on-demand, optimised for querying, and discarded with
edits\cite{roslyn-red-green-trees}, while the ``green'' tree serves as the
source of truth.

As more and more language tools interact with and rely on the compiler, compiler
authors find themselves adapting the pipelined architecture to emit additional
intermediate artifacts. If this evolution happens organically over long periods
of time and without a methodical approach, compiler codebases can degenerate
into clouds of complex, intertwined code.

A possible remedy is to adapt the pipeline architecture in ways that make it
simple to incrementalise and memoize its stages. The key insight is that the
pipeline is conceptually a collection of data-dependent queries. If we express
the compiler in the language of a general query engine, caching and incremental
computation features come for free. Propagating changes through the data
dependency graph is simple, and this change processing subsumes cache
invalidation. The interfaces between compiler queries effectively become
high-level \acrshort{api}s, with little extra work. These interfaces are shared
with other applications, making integration simpler and less error-prone than in
many traditional pipeline architectures, which maintain separate sets of
internal and external interfaces.

\subsubsection*{A functional approach}

Even though the traditional pipeline and its query-based reimagination are
conceptually close, their implementations are vastly different. Typical
compilers mutate global state during the course of a compilation. If any stages
produce additional data, they populate side channels in the form of global maps
and tables for further passes to use. This can lead to subtle compiler bugs when
compiler passes are added, removed, or reordered. Mutable state is also an
obstacle to parallelism and makes it difficult to reason about where exactly in
the pipeline does one intermediate form change into another.

On the other hand, the query-based approach is inherently functional. Compiler
passes are ordered by their data dependencies and the amount of available
parallelism is limited only by the number of independent queries at a given
stage. A query-based compiler should avoid using mutable state, so that the
query engine can correctly propagate changes and update memoized
functions\footnote{Some hidden mutability may still be beneficial for
performance optimisations. For example, a hidden mutable map can serve as a
backing store for string interning. Naturally, the programmer needs to be
vigilant around any such places in the implementation and verify that the
introduced side-effects do not compromise correctness of the entire system.}.

\subsubsection*{Queries in the wild}

Although query-based approaches promise many benefits, their adoption in large
compilers seems scarce. Large compilers and compiler frameworks, including
\acrshort{llvm} and p4c, still build on the traditional pipeline model.

One exception is The Rust compiler. Although it was not originally built around
a query system, one has been retrofitted into it. Major parts of the pipeline
between \texttt{rustc}'s high-level \acrshort{ir} and \acrshort{llvm}
\acrshort{ir} are now implemented as incremental interdependent
queries\cite{rustc-queries}.

Smaller compiler projects have ventured further into the query-based realm of
data dependencies and memoization. Examples can be seen in
Sixten\footnote{\url{https://github.com/ollef/sixten}},
Sixty\footnote{\url{https://github.com/ollef/sixty}}, and
Eclair\footnote{\url{https://github.com/luc-tielen/eclair-lang}}. Olle
Fredriksson developed a dedicated library for query-based build systems dubbed
\texttt{rock}\footnote{\url{https://github.com/ollef/rock}} that all three
projects build on.

%---------------%
\chapter{Design}
%---------------%

The high-level architecture of the \acrshort{p4} Analyzer project marks a
departure from conventional language server designs in that it primarily targets
WebAssembly and aims to run entirely within the Visual Studio Code editor. This
decision makes installation simpler for the end-user, cross-platform support
easier for the developers, and security policy conformance trivial for any
security teams involved.

The main mode of operation is thus as follows: the language server runs in a
WebAssembly worker of the \acrshort{p4} Analyzer VS Code extension. The
extension itself defines a simple TextMate\cite{textmate} grammar specification
and serves as a thin client for the server. The main bulk of \acrshort{lsp}
functionality is delegated to the editor. VS Code forwards edits to open files
to the language server, which updates its model of the workspace. When VS Code
asks for completions, hover, diagnostics, or other features, the analyzer
recomputes necessary information on-demand and responds appropriately.

In addition to the WebAssembly executable, the \acrshort{p4} Analyzer project
also compiles to a native binary that executes in a standalone process and
communicates with an arbitrary \acrshort{lsp}-compliant client over a socket.
However, the standalone language server requires support for certain features
related to filesystem functionality that fall outside the protocol
specification. We will discuss these in-depth later\todo{make sure this is
indeed the case, put a reference here}.

%-----------------------------------------------%
\section{The \pdfacrshort{p4} Analyzer pipeline}
%-----------------------------------------------%

The first step in our pipeline is lexical analysis. Somewhat unconventionally,
our lexer produces tokens even for the preprocessor (i.e. it analyses
preprocessor directives). Our preprocessor then operates at the lexeme level,
rather than running separately as the first step. This requires a
reimplementation of the preprocessor, which is already necessitated by
fault-tolerance and WebAssembly support requirements anyway. On the upside, a
custom preprocessor simplifies tracking of source positions, which are crucial
for accurate diagnostics.

\subsection{Lexical analysis}

The \pfs specification defines a \acrshort{yacc}/Bison grammar for the language.
However, this grammar has several flaws.

For example, it reuses the \texttt{parserTypeDeclaration} non\-terminal in
\texttt{parserDeclaration}s but imposes extra restrictions: a parser declaration
may not be generic. This requires checking the child production outside the
grammar specification.

However, the primary issue is that the grammar design does not maintain a clean
separation between a parser and a lexer and requires these two components to
collaborate.

\begin{displayquote}
	\textit{The grammar is actually ambiguous, so the lexer and the parser must
	collaborate for parsing the language. In particular, the lexer must be able
	to distinguish two kinds of identifiers:}

	\begin{itemize}
		\item \textit{Type names previously introduced (\texttt{TYPE\_IDENTIFIER}
		tokens)}
		\item \textit{Regular identifiers (\texttt{IDENTIFIER} token)}
	\end{itemize}

	\textit{The parser has to use a symbol table to indicate to the lexer how to
	parse subsequent appearances of identifiers.}

	-- \citetitle{p416:v123:spec} \cite{p416:v123:spec}
\end{displayquote}

The specification goes on to show an example where the lexer output depends on
the parser state and mentions that the presented grammar ``\textit{has been heavily
influenced by limitations of the Bison parser generator tool.}''

The tight coupling between the lexer and the parser, as well as the decision to
remain in the confines of an outdated parser generator despite its many
drawbacks, are in our opinion examples of poor design for a language born in the
twenty-first century. We have elected not to follow this ambiguous grammar
specification in the \acrshort{p4} Analyzer project and instead build a pipeline
that is tolerant to invalid input to the fullest extent possible, while
accepting the same language.

Our lexer's task is to convert the input string into a stream of lexemes. The
lexer is a standalone finite state machine independent of any later stages in
the pipeline. It has a secondary output for reporting diagnostics, but this side
channel is write-only.

\subsubsection*{Error tolerance}

Error tolerance at the lexer level means proceeding with lexeme stream
generation despite nonsensical input. We emit a special error token whenever
such input is encountered. Additionally, the lexer validates numeric constants,
which can specify width, base, and signedness. These properties could be out of
bounds for a given literal. In these cases, the lexer should still produce a
valid token while logging an error-level diagnostic. The server can then report
the diagnostic to the user once lexing completes.


\subsection{The preprocessor}

\pfs requires support for a preprocessor, very similar to the C preprocessor,
directly in the specification. However, it does not ask implementors to support
the entirety of \texttt{cpp}. Notably, only simple, parameter-less macros are
allowed. This is already enough to necessitate running the preprocessor before
starting the parser, however. Consider the code in Listing~\ref{lst:p4pp}. These
examples show how grammatically invalid code may become valid and vice versa,
based only on the right-hand sides of preprocessor macros.

An important consideration for a correct implementation of preprocessor
directives is their context-sensitive nature. Expressions for conditional
inclusion in directives \texttt{\#if}, \texttt{\#elif}, and \texttt{\#ifdef} are
themselves subject to macro substitution and thus have to be kept in plain text
or lexeme form until their evaluation.

One more thing to note here is the mechanism of document inclusion. Before
analysing a \pfs source file (at least to some degree), the full extent of its
dependencies is unknown and arbitrary. The language has no module system and
imposes no restriction on the paths a source file can include. This poses a
challenge for lexeme-level preprocessors, as a file needs to be lexed before it
can be included. To deal with this, a correct implementation should collect the
paths a source file can depend on, lex their contents, and include their lexemes
in the preprocessed lexeme stream. This is of particular note in our
implementation, as the collection of dependencies reports this dependency set to
the editor to set up filesystem-level watches. Subsequent edits to the
dependencies, or even to the dependency set itself, can be processed
incrementally.

\subsubsection*{Error tolerance}

Error tolerance in the preprocessor means reporting errors and warnings about
malformed input to the user while continuing to interpret directives in the
input stream on a best-effort basis. Mistakes in preprocessor directives come in
several flavours.

The directive itself may be malformed, either due to a typo in its name or a
problem in some of its arguments. The former case will simply be lexed as an
unrecognized directive and reported as such. It is possible to suggest fixes for
common typos to the user. A problem in the directive's argument or arguments
needs to be resolved based on its meaning. For example, an \texttt{\#include}
directive could point to a non-existent file, the preprocessor should then
report this error and proceed as if the file were not included. This is likely
to lead to further errors down the road, but without knowledge of the referenced
file's contents, it is the best a preprocessor can do.

Another class of errors is semantic and context-sensitive in nature: a directive
may be used in the incorrect context or missing where it is expected. For
example, a user may forget to add an \texttt{\#endif} directive, or include more
than one \texttt{\#else} directive for a condition. Unfortunately, guessing the
user's intention when faced with any syntactic or semantic problems in the input
is a tall order. No guarantees of optimality can be given, as is often the case
with similar heuristics. In the duplicate \texttt{\#else} problem, the
preprocessor could be reasonably expected to either skip over the first
\texttt{\#else}'s body, the second \texttt{\#else}'s body, or assume either of
the directives was inserted by accident and pretend it is not a part of the
input stream. We choose to skip the second \texttt{\#else}'s body in our design,
but other strategies are equally valid.

\begin{lstlisting}[
	caption={~\pfs preprocessor example},
	label=lst:p4pp,
	captionpos=t,
	tabsize=4,
	float,
	abovecaptionskip=-\medskipamount,
	belowcaptionskip=\medskipamount,
	language=p4
]
#define op +
// #define op 2

#define paren )

header h {
	bit<1> field;
}

control pipe(inout h hdr) {
	Checksum16() ck;
	apply {
		// arithmetic expression could be invalid
		h.field = 1 op 3;
		// a parse without prior macro substitution would fail
		ck.clear( paren;
		// this would parse correctly, but macro substitution
		// will reveal a parse error
		ck.update(op);
	}
}
\end{lstlisting}


\subsection{The parser}

The next natural step in the pipeline is the act of finding the productions of a
\pfs grammar that match the preprocessed input program; parsing. While the steps
up to this point are fairly simplistic and efficient, parsing is a
resource-intensive process. A language server is expected to provide real-time
feedback to the developer, including auto-completion suggestions updated with
every keypress. Low latency is crucial to the end-user and the parser lies on
every critical path from user input to high-level results shown in the editor's
interface. At the same time, a typical \pfs program is likely to consist of a
long prefix that does not change between edits and a user-maintained suffix that
changes frequently. This is because a \pfs program usually begins with
\texttt{\#include} directives referencing platform-specific files with
constants, error codes, \extern{} definitions and other shared code. These
constraints and conditions are a very good fit for the field of
\emph{incremental parsing}.

An incremental parser aims to reuse previously computed information about the
input in response to small perturbations. Our parser specifically builds on
incremental packrat parsing\cite{dubroy2017incremental-packrat-parsing}, which
places few constraints on grammar design and is easy to implement in an
extensible manner.

Packrat parsing\cite{ford2002packrat} is a linear time algorithm for recognizing
productions of a \acrfull{peg}. It relies heavily on memoization to avoid costly
backtracking, at the expense of memory overhead.
\linkedciteauthor{dubroy2017incremental-packrat-parsing} augment the packrat
memoization table to support incrementality. The result is a parser that is at
once simple, general, incremental, and efficient.

Our packrat parser conceptually handles \acrlong{peg}s\cite{ford2004parsing}, a
class of unambiguous grammars for context-free languages. \acrshort{peg}s are
syntactically similar to \acrlong{cfg}s. However, the choice operator in
\acrshort{cfg}s is ambiguous, whereas \acrshort{peg}s use ordered choice, which
greedily attempts to match alternatives in order. The right-hand side of a
\acrlong{peg} rule can also contain predicates, which attempt to match without
consuming input. Predicates are useful for positive and negative look\-ahead.

\begin{figure}[h]
	\centering
	\begin{align*}
		e ::= & \hspace{2pt} \varepsilon & \text{(empty string)} \\
		\mid  & \hspace{2pt} \texttt{t}  & \text{(terminal)} \\
		\mid  & \hspace{2pt} e_1 e_2     & \text{(sequence)} \\
		\mid  & \hspace{2pt} e_1 | e_2   & \text{(ordered choice)} \\
		\mid  & \hspace{2pt} e^*         & \text{(zero or more)} \\
		\mid  & \hspace{2pt} e^+         & \text{(one or more)} \\
		\mid  & \hspace{2pt} e?          & \text{(zero or one)} \\
		\mid  & \hspace{2pt} \&e         & \text{(positive lookahead)} \\
		\mid  & \hspace{2pt} !e          & \text{(negative lookahead)} \\
		\texttt{t} \in & \hspace{2pt} \text{tokens}
	\end{align*}
	\caption{Syntax of \acrlong{peg}s.}
	\label{fig:peg-syntax}
\end{figure}

The syntax of typical \acrlong{peg}s can be seen in Figure~\ref{fig:peg-syntax}.
Our grammars are slightly simpler: we implement neither positive lookahead nor
the $+$ and $?$ operators. Positive lookahead can be simulated by nesting two
negative lookaheads, and the $+$ and $?$ operators can desugar to combinations
of general repetition, ordered choice, and the empty string. These decisions
were made to simplify the parser implementation, but in turn complicate the
grammar with verbose and repetitive definitions. It remains to be seen whether
they survive future refactorings. A further restriction is that all grammar
rules must contain at most one level of nesting, i.e. the grammar must be given
in a normal form where the immediate subtrees of a rule's right-hand side are
all non-terminals.

Our design differentiates between a generic parser library and a parser built on
it. Grammars are defined using a small \acrshort{dsl} implemented with Rust's
declarative macros. An example can be seen in
Listing~\ref{lst:grammar-dsl-example}. The~\texttt{grammar!} macro expands to a
data structure representing the grammar itself. This structure can be passed to
a smart constructor, which validates the grammar\footnote{Ensuring all
referenced non-terminals are in fact defined, and that \texttt{start} is
present.} and returns a parser. The implementation interprets the grammar
\acrshort{dsl} at runtime.

If future testing and development necessitate optimization of the parser, there
is room to build a parser compiler for the \acrshort{dsl} and generate a more
efficient solution from the same grammar. Procedural Rust macros\footnote{While
declarative macros can only perform a very restricted set of rewriting
operations on token trees, procedural macros can run arbitrary Rust code during
expansion.} could take care of integrating the parser compiler into the build
process.

\subsubsection*{Incremental updates}

Since the parser is required to process incremental updates to the input
sequence, it is not simply a function from a sequence of tokens to a parse tree.
Rather, the parser takes a reference to a read-write lock of the input. It
defines an \texttt{apply\_edit} method that acquires a write lock of the input
sequence, applies the change, invalidates relevant entries in the memoization
table, and releases the lock.

To initiate parsing, the user invokes the \texttt{parse} method, which acquires
a read lock on the input for the duration of parsing.


\begin{lstlisting}[
	caption={~Example grammar in our \acrshort{dsl}.},
	label=lst:grammar-dsl-example,
	captionpos=t,
	tabsize=4,
	float,
	abovecaptionskip=-\medskipamount,
	belowcaptionskip=\medskipamount,
	language=c
]
grammar! {
	// The initial non-terminal is called `start`
	start => p4program;
	// The postfix `rep` operator corresponds to Kleene star
	ws => whitespace rep;
	// Non-terminals can expand to terminals by wrapping the terminal in
	// parentheses
	whitespace => (Token::Whitespace);

	// The right hand side can also be a sequence separated by commas
	p4program => ws, top_level_decls, ws;
	// ...or a choice separated by pipes
	top_level_decls =>
		top_level_decls_rep | top_level_decls_end | nothing;
	top_level_decls_rep => top_level_decl, ws, top_level_decls;
	top_level_decls_end => (Token::Semicolon);

	direction => dir_in | dir_out | dir_inout;
	// Rules can also match tokens against an arbitrary Rust pattern,
	// which is useful for identifying soft keywords
	dir_in    => { Token::Identifier(i) if i == "in" };
	dir_out   => { Token::Identifier(i) if i == "out" };
	dir_inout => { Token::Identifier(i) if i == "inout" };
}
\end{lstlisting}


\subsubsection*{Error reporting}

Error handling in parsing is far more nuanced than in any of the previous steps,
which is not surprising, considering the relative complexity of the languages
that the individual steps recognize and process. A good parser should attempt to
provide as much feedback as possible to the user, even when faced with
unexpected tokens. It is not enough to simply stop at the first error, and it is
not enough to be imprecise about the locations of problems in the source file.
Both of these considerations pose some challenges.

The desire to continue parsing malformed input to provide feedback to the
developer has a long history\cite{graham1975practical}. Error recovery has been
studied at length in \acrlong{peg}s as well\cite{redziejowski2009mouse,
maidl2013exception, demedeiros2016parsing, demedeiros2018syntax,
demedeiros2020automatic}. The \acrshort{peg} case is interesting, because a
packrat parser relies on failures to guide choice selection. By the time a
parsing failure propagates to the starting non-terminal, the information about
the context that led to it is lost.

Specifically, when encountering unexpected input, a packrat parser unwinds the
stack to a ``calling'' ordered choice operator, and attempts to parse the next
alternative. The next alternative is likely the wrong choice, however. The
recently failed alternative should have matched, but encountered invalid input.
Thus, many other alternatives may fail before an error eventually propagates to
the user. The end result is that the programmer receives an unhelpful error
message that could potentially come from a position many tokens before the
actual problem's origin. This specific challenge has a popular practical
solution in the form of the \emph{farthest failure
heuristic}\cite{ford2002packrat-non-func}. It is based on the observation that
the alternative that should have matched will probably process the longest
prefix of the token stream.

While the farthest failure heuristic addresses the location problem in many
practical situations, it is not a general solution. Worse, the imprecise error
reporting of \acrshort{peg}s has other implications as well. Notably, a common
consideration for error messages is the suggestion of expected tokens to the
user, to provide a rudimentary selection of possible fixes. However, the
compounding failures of \acrshort{peg} choice operators grow the set of expected
tokens, which makes the suggestions in error messages irrelevant.

Intuitively, the problem with \acrshort{peg} error reporting is that
non-terminals deeper down the parse tree have no way of distinguishing between
an error in the input that will ultimately cause the overall parse to fail, and
an error that can be recovered from in an ordered choice operator higher up.

To address this problem, \linkedciteauthor{maidl2013exception} conservatively
extended the \acrshort{peg} formalism with \emph{error
labels}\cite{maidl2013exception}. Error labels semantically\footnote{In the
sense that we are adding semantic information to the process that analyzes
syntax.} stand for errors that a grammar rule can raise when encountering
unexpected input. This differentiates errors caused by nonsensical input from
benign parse failures, and thus solves the problem of accurate error reporting
in \acrlong{peg}s\footnote{At the expense of extensive manual annotations of the
grammar. See \cite{demedeiros2020automatic} for a possible remedy.}. The
extension comprises several components:

\begin{itemize}
	\item The \acrlong{peg} is extended with a finite set of labels $L$.
	\item A special failure label \texttt{fail}, $\texttt{fail} \not \in L$, is
	also added to the grammar. This label indicates a benign failure that can be
	caught by the ordered choice operator.
	\item The grammar of \acrlong{peg} right-hand sides is enriched with the
	\texttt{throw} operator, which takes a label $l \in L$ as an argument. An
	error thrown by \texttt{throw} cannot be caught by ordered choice and thus
	indicates a parse error that should be immediately reported to the user.
	\item Rules are modified to include instances of the \texttt{throw}
	operator.
\end{itemize}

The error label extension is reminiscent of exception handling in ordinary
programming languages, and indeed was originally modelled after it, complete
with an extension of ordered choice playing (the role of)
\texttt{catch}\cite{demedeiros2016parsing}. However, the \texttt{catch}-like
mechanism is not necessary for error recovery, so we will not discuss it
further.

Error labels returned by the parsing process can be mapped to readable error
messages. Because parsing terminates early, the set of expected tokens is kept
accurate. In addition, having a single, obvious point of failure also makes
location tracking trivial.

\subsubsection*{Tracking source locations}

Tracking precise source locations is a common requirement for compilers and is
arguably even more important for language servers. Our lexer provides the
precise span of source code for every token and our preprocessor keeps track of
where a token came from during file inclusion. It is important to track the
entire ``inclusion path'' for every token, which consists of segments of file
identifiers and spans, since a single file can be included multiple times from
multiple locations, with or without detours through other files.

The consideration for the parser here is twofold, namely, to continue to track
the source locations of tokens and grammar productions in the format provided by
the preprocessor, and to maintain incrementality while doing so.

The first consideration requires some care in pattern-matching of tokens -- the
grammar \acrshort{dsl} is not intended to pattern-match on token inclusion
paths. Additionally, an important question is how to assign inclusion paths to
grammar productions. This is trivial for terminals, lookahead, and alterations.
The interesting case is how to handle sequences and repetitions. While a grammar
production could, in theory, store the set of all inclusion paths that it
depends on, doing so is wasteful and not very helpful. Instead, we can inspect
the inclusion paths of the first and the last token in the sequence or
repetition. If these two paths match (meaning they differ at most in the last
span), we can simply use this path, adjusting the final span. This loses
information about tokens from the middle of the sequence, but note that subtrees
will still include it. If the two paths do not match, we can take their longest
common suffix, again adjusting the last span\footnote{Note that our
implementation is incomplete in this regard and presently only reports token
spans local to the parsed file.}.

To maintain incrementality, the parser needs to reuse previous parse results in
response to small changes of the input. Using actual token spans would be
problematic, since a single character change could invalidate the spans of
potentially all tokens in the file. Instead, the parser needs to work with the
relative notion of token count. Every \acrshort{cst} node stores the number of
tokens that it spans (the number of tokens in its subtree). A traversal of the
\acrshort{cst} can then reconstruct the absolute offset of each node without
ever explicitly storing it.

One more thing to note here is that the units of offsets in the \acrshort{cst},
and therefore of the reconstructed spans, are individual tokens. These need to
be converted to spans in the source text before being reported to the user.

\subsection{Abstract syntax trees}

The presented packrat parser produces a \emph{\acrfull{cst}}. A \acrlong{cst} is
a full-fidelity representation of a grammar production, meaning it includes all
tokens and non-terminals, including comments, whitespace, and intermediate
non-terminals introduced to circumvent the constraints of the grammar
\acrshort{dsl}. Preserving full-fidelity syntax trees is important to maintain
the parser's incremental performance and avoid expensive backtracking, but
\acrshort{cst}s include a lot of state that is unimportant to further processing
and analysis steps.

Moreover, the shape and types of these trees are determined by the structure of
the grammar. That is, nodes in a \acrshort{cst} are alterations, repetitions,
sequences, etc. Analysis code would instead prefer a typed \acrshort{api} to
access the semantic, abstract nodes in the syntax tree, without interference
from trivia tokens and non-terminals.

\subsubsection*{Tree abstractions}

These use cases are covered by \acrshort{ast} translation. The \texttt{ast}
module provides three layers of abstraction over \acrshort{cst}s:
\texttt{GreenNode}s, \texttt{SyntaxNode}s, and \texttt{AstNode}s.

A \texttt{GreenNode} wraps a successful parsing result (an
\texttt{ExistingMatch<P4GrammarRules, Token>}) in a reference-counted cell, to
simplify reuse. It provides a \texttt{children} method that returns an iterator
over the children of a node. This iterator is a heap-allocated \texttt{dyn} type
to smooth over differences in the backing \acrshort{cst} node\footnote{Future
extension and optimisation efforts may decide to change the \acrshort{cst}
representation to avoid such indirections in tree traversals.}. The children are
\texttt{GreenNode}s themselves, wrapped transparently by the iterator.
\texttt{GreenNode}s form immutable, functional trees. They are cheap to clone
(thanks to reference counting) and can be structurally shared, although not
between threads\footnote{This could be amended by using atomic reference
counting cells instead.}.

A \texttt{SyntaxNode} is a zipper data structure that maintains a parent pointer
(and therefore the path to the root of the tree) along with the absolute offset
in tokens. It again provides a \texttt{children} method with an iterator that
yields \texttt{SyntaxNode}s. These zippers are useful for traversing
\texttt{GreenNode}s. A \texttt{SyntaxNode} also carries a reference to the
grammar definition and keeps track of the non-terminal that it is in, by storing
it alongside the parent pointer. This is important, as neither the
\texttt{GreenNode} nor the \acrshort{cst} provide this information.

Finally, an \acrshort{ast} node is a typed wrapper around a \texttt{SyntaxNode}.
Unlike the previous two types, it is not a single \texttt{struct}. Instead,
there is a different type for each \acrshort{ast} node, but all of them share a
common interface via the \texttt{AstNode} trait. This trait provides methods to
cast a \texttt{SyntaxNode} to a particular \texttt{AstNode} type, to access the
backing \texttt{SyntaxNode}, and to intuit the node's offset and span.

Many \acrshort{ast} nodes share the binary representation with their underlying
\texttt{SyntaxNode}s, meaning they only have a single field of the
\texttt{SyntaxNode} type. Additionally, the implementations of the
\texttt{AstNode} trait are often structurally similar. To avoid code duplication
and simplify maintenance, the \texttt{ast} module provides the
\texttt{ast\_node!} macro, through which most \acrshort{ast} nodes are defined.

\begin{lstlisting}[
	language=Rust,
	tabsize=4,
	float,
	caption={The signature of the \texttt{ast\_node!} macro.},
	label={lst:ast_node_macro},
]
macro_rules! ast_node {
	($non_terminal:ident $(, methods: $($method:ident),+)?) => {
		paste! {
			// see later listings for the body of this macro
		}
	};
}
\end{lstlisting}

The signature and top level of the macro are shown in
Listing~\ref{lst:ast_node_macro}. Note the inclusion of the \texttt{paste!}
block: the \texttt{paste} crate\footnote{\url{https://crates.io/crates/paste}}
provides means for converting the case of identifiers. We use this throughout
the \texttt{ast\_node!} macro to generate the \texttt{struct} name from the name
of the non-terminal. Rust types are conventionally written in camel case while
methods, as well as our non-terminals, have snake case names.

The body of the macro defines the \texttt{struct} and implements the
\texttt{AstNode} trait for it. To generate documentation comments, the macro's
body gives \acrshort{ast} node \texttt{struct}s the \texttt{doc} attribute. The
line can be seen below, it was omitted from
Listing~\ref{lst:ast_node_macro_body} to avoid confusing the lexer:

\texttt{\#[doc = "AST node for [`P4GrammarRules::" \$non\_terminal "`]."]}

\begin{lstlisting}[
	language=Rust,
	tabsize=4,
	float,
	caption={The main body of the \texttt{ast\_node!} macro generates newtypes for \texttt{SyntaxNode}s and implements \texttt{AstNode} for them.},
	label={lst:ast_node_macro_body},
]
#[derive(Debug, Clone, PartialEq, Eq, PartialOrd, Ord, Hash)]
pub struct [<$non_terminal:camel>] {
	syntax: SyntaxNode,
}

impl AstNode for [<$non_terminal:camel>] {
	fn can_cast(node: &SyntaxNode) -> bool {
		node.kind() == P4GrammarRules::$non_terminal
	}

	fn cast(node: SyntaxNode) -> Option<Self> {
		if node.kind() == P4GrammarRules::$non_terminal {
			Some(Self { syntax: node })
		} else {
			None
		}
	}

	fn syntax(&self) -> &SyntaxNode { &self.syntax }
}

// continues with optional methods
\end{lstlisting}

Finally, the invocation of \texttt{ast\_node!} can optionally contain a list of
methods to implement on the \texttt{struct} itself, outside the scope of the
\texttt{AstNode} trait. The implementation of this functionality can be seen in
Listing~\ref{lst:ast_node_macro_body_methods}. These generated methods provide
type-safe access to children while transparently handling both the skipping of
trivia nodes and invalid underlying \acrshort{cst}s. Each returns an iterator
without a guarantee of exactly how many children will be iterated over,
smoothing over issues of arity. It is up to the users of this \acrshort{api} to
detect and report such errors.

\begin{lstlisting}[
	language=Rust,
	tabsize=4,
	float,
	caption={The optional methods section of the \texttt{ast\_node!}
	macro's body.},
	label={lst:ast_node_macro_body_methods},
]
$(impl [<$non_terminal:camel>] {
	$(
		#[doc = "Fetch the `" $method "` child of this node."]
		pub fn $method(
			&self
		) -> impl Iterator<Item = [<$method:camel>]> {
			fn go(
				node: SyntaxNode
			) -> Box<dyn Iterator<Item = SyntaxNode>> {
				match node.trivia_class() {
					TriviaClass::SkipNodeAndChildren =>
						Box::new(std::iter::empty()),
					TriviaClass::SkipNodeOnly =>
						Box::new(node.children().flat_map(go)),
					TriviaClass::Keep =>
						Box::new(std::iter::once(node)),
				}
			}

			self.syntax()
				.children()
				.flat_map(go)
				.filter_map([<$method:camel>]::cast)
		}
	)+
})?
\end{lstlisting}

The entire \acrshort{ast} translation layer was inspired by the internals of
rust-analyzer\footnote{\url{https://rust-analyzer.github.io/}}, an
\acrshort{lsp}-compliant language server for Rust. While rust-analyzer's syntax
tree manipulations work with a different underlying representation and rely on a
recursive descent parser, the high-level interfaces, the split between
\texttt{GreenNode}s, \texttt{SyntaxNode}s, and \texttt{AstNode}s, as well as the
heavy reliance on code generation, are all motivated directly by rust-analyzer's
developer documentation. We would like to thank the rust-analyzer developers for
maintaining a comprehensive high-level documentation of their approach and
including known alternative approaches, e.g. from Roslyn and IntelliJ, that
other developers can learn from.

\subsubsection*{Identifying trivia nodes}

The reader will observe that we have carelessly included a previously
undiscussed type in the \texttt{ast\_node!} macro's body: the
\texttt{TriviaClass} \texttt{enum}.
Listing~\ref{lst:ast_node_macro_body_methods} makes it quite clear that the
\texttt{TriviaClass} type determines which subtrees to iterate over when
invoking a generated method of an \texttt{AstNode} type.

Our grammar \acrshort{dsl} has a few extra features. The author of a grammar can
identify the trivia non-terminals by including \emph{annotations} in the grammar
definition. Listing~\ref{lst:trivia_annotation} shows how to use these
annotations. By default, every non-terminal has a \texttt{TriviaClass} of
\texttt{Keep}, which means it will be enumerated by an \texttt{AstNode}'s
iterator.

It is worth noting that the macro for defining the \acrshort{p4} grammar is also
the source of truth for the set of non-terminals: it simultaneously defines the
\texttt{P4GrammarRules} \texttt{enum}. This is very useful for developers, since
it avoids code duplication and enables features like go-to definition or find
references right in the grammar definition. Using an \texttt{enum} to represent
non-terminals is also a requirement for exhaustiveness checking, in case any
later steps in the pipeline require it. The grammar \acrshort{dsl} forwards
documentation comments above productions to the variants of
\texttt{P4GrammarRules}.

\begin{lstlisting}[
	language=Rust,
	tabsize=4,
	float,
	caption={An example of trivia annotations and doc comments in the grammar
	\acrshort{dsl}.},
	label={lst:trivia_annotation},
]
grammar! {
	@SkipNodeAndChildren at_symbol close_paren;
	@SkipNodeOnly maybe_direction parameter_comma;

	/// Semantic non-terminal that marks an identifier as a definition.
	///
	/// For example, in `parser MyParser<T>(inout T x) { }`,
	/// `MyParser`, `T`, and `x` are all definitions,
	/// and possible targets for go-to definition.
	definition => ident;

	// other non-terminals omitted
}
\end{lstlisting}


\listoftodos[Work in progress]
 % include `text.tex' from `text/' subdirectory

\appendix\appendixinit % do not remove these two commands

\chapter{Sample appendix}

Whatever doesn't belong to the main part should be put here.
 % include `appendix.tex' from `text/' subdirectory

\backmatter % do not remove this command

\printbibliography % print out the BibLaTeX-generated bibliography list

\include{text/medium} % include `medium.tex' from `text/' subdirectory

\end{document}
