% Do not forget to include Introduction
%---------------------------------------------------------------
% \chapter{Introduction}
% uncomment the following line to create an unnumbered chapter
\chapter*{Introduction}\addcontentsline{toc}{chapter}{Introduction}\markboth{Introduction}{Introduction}
%---------------------------------------------------------------
\setcounter{page}{1}

% The following environment can be used as a mini-introduction for a chapter.
% Use that any way it pleases you (or comment it out). It can contain, for
% instance, a summary of the chapter. Or, there can be a quotation.
\begin{chapterabstract}
	TODO: abstract
\end{chapterabstract}

Programming Protocol-independent Packet Processors (P4) is a domain-specific
language for programming network switches. Since its introduction in
2014\cite{p4original}, it became widely popular in software-defined networking
(SDF).




Over the last decade, \textit{language servers} became a popular architecture
for providing semantically informed editing features in integrated developer
environments and lightweight source code editors alike. Although examples of
language server -like tools predate their standardization, a major milestone in
their development was the introduction of the Language Server Protocol (LSP).
% TODO: make that into an actual acronym
The support of (LSP) in Visual Studio Code seeded an ever-growing environment of
cross-platform tooling for a variety of programming, configuration,
specification, and markup languages. Their success can be attributed in part to
the investment by Microsoft. However, (LSP) as a standard has much technical
merit.

Before the introduction of a common communication interface between
language-specific tooling and a given source code editor, implementors had to
create and maintain extensions for any number of different environments, often
in different programming languages. For example, take an editor plugin providing
smart completion and "go to definition" features for the Python programming
language. Suppose the author aims to extend Vim, Emacs, and IntelliJ IDEA. They
would therefore have to reimplement the given functionality in vimscript, Emacs
Lisp, and a JVM-based language. Even though recent innovations in classical text
editors(Neovim) have partly moved away from using their own (DSL)'s, the author
is still burdened with three times the implementation work, as every development
environment (API) is different.

The solution to this problem is to split editor extensions according to the
client-server model. The major implementation work for domain-specific
functionality resides on the server side, whereas the client is only a
relatively thin wrapper that adapts a given editor's API. Implementors can build
many thin clients for different development environments that all connect to the
same \textit{language server}. (LSP)'s raison d'être is support of this model in
Visual Studio Code.

Codenvy, Microsoft, and Red Hat collaborated on standardizing the protocol to
enable other tools % TODO: finish this
% TODO: https://sdtimes.com/che/codenvy-microsoft-red-hat-collaborate-language-server-protocol/
%       https://www.infoworld.com/article/3088698/microsoft-backed-langauge-server-protocol-strives-for-language-tools-interoperability.html

%---------------------------------------------------------------
\section{Outline}
%---------------------------------------------------------------

\begin{enumerate}
  \item (intro) P4 information
  \item (intro) language servers, VS Code (screenshots, c++?)
  \item design: overview of the architecture, interfaces with the editor, what
        needs to be implemented
  \item subsections about problems that needed solving, but most should go into
        the design part!
  \item results, what it can do, open-source repository, performance metrics
        (aim for interactivity)
  \item conclusion, next steps
\end{enumerate}


authorship: link to GitHub, detail which non-critical parts were implemented by
a colleague


\begin{lstlisting}[caption={~Zbytečný kód},label=list:8-6,captionpos=t,float,abovecaptionskip=-\medskipamount,belowcaptionskip=\medskipamount,language=C]
    #include<stdio.h>
    #include<iostream>
    // A comment
    int main(void)
    {
        printf("Hello World\n");
        return 0;
    }
\end{lstlisting}

%%%%%%%%%%%%%%%%%%%%%%%%%%%%%%%%%
% alternative using package minted for source highlighting
% package minted requires execution with `-shell-escape'
% e.g., `xelatex -shell-escape ctufit-thesis.tex'
% \begin{listing}
% \caption{Zbytečný kód}\label{list:8-6}
% \begin{minted}{C}
%     #include<stdio.h>
%     #include<iostream>
%     // A comment
%     int main(void)
%     {
%         printf("Hello World\n");
%         return 0;
%     }
% \end{minted}
% \end{listing}
% %%%%%%%%%%%%%%%%%%%%%%%%%%%%%%%%%
Nullam feugiat, turpis at pulvinar vulputate, erat libero tristique tellus, nec
bibendum odio risus sit amet ante. Aenean id metus id velit ullamcorper
pulvinar. Fusce wisi. Integer lacinia. Aliquam id dolor. Pellentesque pretium
lectus id turpis. Suspendisse sagittis ultrices augue. In laoreet, magna id
viverra tincidunt, sem odio bibendum justo, vel imperdiet sapien wisi sed
libero. Sed ac dolor sit amet purus malesuada congue. \cite{Crochemore2002}

\begin{table}\centering
\caption[Příklad tabulky]{~Zadávání matematiky}\label{tab:matematika}
\begin{tabular}{l|l|c|c}
	Typ		& Prostředí		& \LaTeX{}ovská zkratka	& \TeX{}ovská zkratka	\tabularnewline \hline
 	Text		& \verb|math|		& \verb|\(...\)|	& \verb|$...$|	\tabularnewline \hline
 	Displayed	& \verb|displaymath|	& \verb|\[...\]|	& \verb|$$...$$|	\tabularnewline
\end{tabular}
\end{table}


%---------------------------------------------------------------
\chapter{The P4 Language}
%---------------------------------------------------------------

\begin{chapterabstract}
	\dots in which we delve into the syntax and semantics of P4, explain its use
	cases, and discuss the differences to conventional programming languages.
\end{chapterabstract}

P4 is a high-level language for programming packet processors.



\section{Donec odio tempus molestie}

\lipsum[2] \cite{def:1, def:2}

\subsection{Class aptent taciti}

\lipsum[2-3]

\begin{description}
\item[Kapitola 1] Lorem ipsum dolor sit amet, consectetuer adipiscing elit.
Curabitur sagittis hendrerit ante. Class aptent taciti sociosqu ad litora
torquent per conubia nostra, per inceptos hymenaeos. Cras pede libero, dapibus
nec, pretium sit amet, tempor quis.

\item[Kapitola 2] Lorem ipsum dolor sit amet, consectetuer adipiscing elit.
Curabitur sagittis hendrerit ante. Class aptent taciti sociosqu ad litora
torquent per conubia nostra, per inceptos hymenaeos. Cras pede libero, dapibus
nec, pretium sit amet, tempor quis.

\item[Kapitola 3] Lorem ipsum dolor sit amet, consectetuer adipiscing elit.
Curabitur sagittis hendrerit ante. Class aptent taciti sociosqu ad litora
torquent per conubia nostra, per inceptos hymenaeos. Cras pede libero, dapibus
nec, pretium sit amet, tempor quis.

\item[Kapitola 4] Lorem ipsum dolor sit amet, consectetuer adipiscing elit.
Curabitur sagittis hendrerit ante. Class aptent taciti sociosqu ad litora
torquent per conubia nostra, per inceptos hymenaeos. Cras pede libero, dapibus
nec, pretium sit amet, tempor quis.
\end{description}

\lipsum[2]

\section{Lorem ipsum dolor sit amet}

\lipsum[3-5]
