% Do not forget to include Introduction
%---------------------------------------------------------------
% \chapter{Introduction}
% uncomment the following line to create an unnumbered chapter
\chapter*{Introduction}\addcontentsline{toc}{chapter}{Introduction}\markboth{Introduction}{Introduction}
%---------------------------------------------------------------

% The following environment can be used as a mini-introduction for a chapter.
% Use that any way it pleases you (or comment it out). It can contain, for
% instance, a summary of the chapter. Or, there can be a quotation.
\begin{chapterabstract}
	\dots in which we get to know the context that gave rise to the
	\acrshort{p4} language and the challenges we set out to overcome.
\end{chapterabstract}

\acrfull{p4} is a domain-specific language for programming network switches. Its
release started a shift in the field of \acrfull{sdn} which, up to that point,
relied heavily on fixed-function hardware for high-performance networking
applications. Similarly to how the C language became a de facto portable
assembler, abstracting over the details of each microprocessor, \acrshort{p4}
abstracts over the details of network processors by presenting a deeply
customizable interface shared by networking software and hardware alike.

Unlike previous approaches in \acrshort{sdn}, \acrshort{p4} does not have
built-in support for common network protocols like TCP, IP, or Ethernet.
Instead, it provides protocol-independent constructs that users can leverage in
order to define arbitrary protocols and instruct a flexible network switch on
how to handle them. This programmability lets a network engineer specify the
configuration and packet processing steps of a router architecture independently
of the underlying machinery.

The newfound flexibility of network processor programming that \acrshort{p4}
enables does not get a chance to shine on traditional network hardware, which is
built for a predetermined set of protocols and processing functions. The real
power of \acrshort{p4} is unlocked by \emph{programmable} network processors,
which can be reconfigured to support novel protocols and forwarding setups. The
commercial sector answered the call for such hardware, for example in Intel's
Tofino line of
chips\footnote{\url{https://www.intel.com/content/www/us/en/products/network-io/programmable-ethernet-switch.html}}.

\acrshort{p4} became wildly popular in \acrshort{sdn} since its introduction in
2014\cite{p4original}, sparking both research and commercial applications. Three
years later, \acrshort{p4} underwent a major redesign\cite{p416}, which
simplified the syntax and removed special-purpose language constructs in favour
of more general solutions\footnote{Specifically, features like counters and
checksum units were replaced by \extern{}s, a universal construct for specifying
additional hardware capabilities not explicitly covered by the core syntax. The
language shrank from over 70 to less than 40
keywords\cite{p416:v1:spec:comparison}.}. The redesigned language is known as
\pfs\cite{p416:v123:spec} and its specification has since received more
incremental updates.

%-------------------------------%
\section*{A Growing Community}
%-------------------------------%

\acrshort{p4} gave rise to a varied ecosystem of commercial offerings of both
hardware and software. It spawned several academic projects that investigated
its semantics\cite{doenges2021petr4}, allocation to heterogeneous
hardware\cite{sultana2021flightplan}, open-source network
testing\cite{antichi2014osnt}, and sketch-based
monitoring\cite{namkung2022sketchlib}, among many
others\cite{liatifis2023advancingp4survey}. The \acrfull{onf}
maintains\cite{p4onf} an open-source reference implementation of a
\pfs\footnote{Although primarily developed for \pfs since the revision of the
language, it is also capable of migrating P4\textsubscript{14} programs to \pfs
or directly compiling them.} compiler frontend and mid\-end, along with several
backends:

\begin{description}
	\item[p4c-bm2-ss] Targets a sample software switch for testing purposes.
	\item[p4c-dpdk] Targets the DPDK software switch (SWX) pipeline\cite{dpdkDPDKRelease}.
	\item[p4c-ebpf] Generates C code which can be compiled to eBPF
		and then loaded in the Linux kernel.
	\item[p4test] A source-to-source P4 translator for testing, learning compiler
		internals and debugging.
	\item[p4c-graphs] Generates visual representations of P4 programs.
	\item[p4c-ubpf] Generates eBPF code that runs in user-space.
	\item[p4tools] A platform for P4 test utilities, includes a test-case
		generator for P4 programs.
\end{description}

All of these components are open source. The frontend and mid\-end of the
reference compiler provide a foundation for hardware vendors to support
\acrshort{p4} in their products and serve the community in resolving
discrepancies between commercial compilers and the language specification.

Despite this growth, \acrshort{p4} has only mediocre support for real-time
feedback to the programmer -- the vast majority of open and commercial tools
rely on compiler output to provide semantic insight into a P4
program\cite{p4insight}, or, as evidenced by the open backends, are parts of the
compiler proper. This is a problem, because the compiler was not designed for
interactive use. Compilations of complex programs can take over an hour to
complete. Long feedback loops hamper development. What's more, information
provided by the compiler frontend is very limited -- the compiler typically
reports at most one error message with very little (if any) explanation. The
lack of an integration between the compiler and development environments is an
obstacle to new users and a neglected area of \acrshort{p4} developer
experience. Even an ergonomic presentation of error and warning messages would
be a large improvement.

We can do far better, as interactive editing support for conventional
programming languages clearly demonstrates. Integrated developer environments
provide many features that \acrshort{p4} programmers can only dream of,
including autocompletion, documentation pop-ups, real-time diagnostics, code
navigation, automatic formatting, and refactoring. These features are not only
convenient, but also improve the quality of software by reducing the cognitive
load on the programmer.

%-----------------------------------------%
\section*{Language Servers to the Rescue}
%-----------------------------------------%

Over the last decade, \emph{language servers} became a popular
architecture\cite{barros2022editing} for providing semantically informed editing
features in integrated developer environments and lightweight source code
editors alike. Although examples of language server -like tools predate their
standardization\cite{bour2018merlin}, a major milestone in their development was
the introduction of the \acrfull{lsp}. The support of \acrshort{lsp} in Visual
Studio Code seeded an ever-growing environment of cross-platform tooling for a
variety of programming, configuration, specification, and markup languages.
Their success can be attributed in part to the investment by Microsoft. However,
\acrshort{lsp} as a standard has much technical merit.

Before the introduction of a common communication interface between
language-specific tooling and a given source code editor, implementors had to
create and maintain extensions for any number of different environments, often
in several programming languages. For example, take an editor plugin providing
smart completion and "go to definition" features for the Python programming
language. Suppose the author aims to extend Vim, Emacs, and IntelliJ IDEA. They
would therefore have to reimplement the given functionality in vimscript, Emacs
Lisp, and a JVM-based language. Even though recent innovations in classical text
editors\footnote{One such project is a fork and refactoring of Vi IMproved,
\emph{Neovim}: \url{https://github.com/neovim/neovim}.} have partly moved away
from using their own \acrshort{dsl}'s, the author is still burdened with three
times the implementation work, as every development environment \acrshort{api}
is different.

The solution to this problem is to split editor extensions according to the
client-server model. The major implementation work for domain-specific
functionality resides on the server side, whereas the client is only a
relatively thin wrapper that adapts a given editor's API. Implementors can build
many thin clients for different development environments that all connect to the
same \emph{language server}. \acrshort{lsp}'s raison d'être is support of this
model in Visual Studio Code.

Coden\-vy, Microsoft, and Red Hat collaborated on standardizing the protocol to
enable other tools to benefit from a shared
ecosystem\cite{sdtimesCodenvyMicrosoft,infoworldMicrosoftbackedLanguage}.

%-----------------------------------------%
\section*{Combining the two}
%-----------------------------------------%

With these developments in mind, we would like to embark on the journey of
supporting the \acrshort{p4} programmer in their development efforts. By
building a language server, we aim to bridge the tooling gap and bring real-time
feedback into the \acrshort{p4} programmer's development environment.

%-----------------------------------------%
\subsection*{A note about authorship}
%-----------------------------------------%

This work concerns the \acrshort{p4} Analyzer project, designed and developed
primarily by Timothy Roberts and me\footnote{With recent help from Andrew Foot
and Yusuf Adam.}, as a part of my internship at Intel. Tim was largely in charge
of managing the integration into the Visual Studio Code editor, the server's
compliance with the \acrlong{lsp}, and the networking aspects of the
project\footnote{This involved unhealthy doses of both asynchronous code and
Rust borrow checker errors, for which I can only offer sincere thanks and
condolences. Tim somehow managed to stay optimistic and kind throughout. He
rocks!}. I was responsible for the implementation of the analyzer core, which
involved the lexing, preprocessing, and parsing of \acrshort{p4} code, designing
the incremental computation architecture of the project, the interfaces the
analyzer core exposes, and the abstractions used within.

To avoid problems of authorship, this thesis focuses on the analyzer core, which
is an isolated module that I was largely in charge of. I can honestly claim
that, at the time of writing, all the code in the \texttt{analyzer-core} crate
is my own. On the other hand, \acrshort{p4} Analyzer is a large project with
even larger ambitions. Tim and I were both involved in large design decisions.
