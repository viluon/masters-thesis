% Do not forget to include Introduction
%---------------------------------------------------------------
% \chapter{Introduction}
% uncomment the following line to create an unnumbered chapter
\chapter*{Introduction}\addcontentsline{toc}{chapter}{Introduction}\markboth{Introduction}{Introduction}
%---------------------------------------------------------------

% The following environment can be used as a mini-introduction for a chapter.
% Use that any way it pleases you (or comment it out). It can contain, for
% instance, a summary of the chapter. Or, there can be a quotation.
\begin{chapterabstract}
	\todo[inline]{abstract}
\end{chapterabstract}

\acrfull{p4} is a domain-specific language for programming network switches. It
became widely popular in \acrfull{sdn} since its introduction in
2014\cite{p4original}, and has been described as the \emph{natural evolution}
of OpenFlow, a standard communications protocol for \acrshort{sdn}.
\acrshort{p4} lets a network engineer specify the data plane\footnote{Also known
as the ``forwarding plane."} of a router architecture independently of the
underlying hardware. A \acrshort{p4} compiler can then synthesise microcode for
a given switch platform and generate library files for the control plane. Unlike
previous approaches in \acrshort{sdn}, \acrshort{p4} is does not have built-in
support for common network protocols like TCP, IP, or Ethernet. Instead, the
language provides protocol-independent constructs that users can leverage in
order to define arbitrary protocols and instruct a flexible network switch on
how to handle them.

In 2017, \acrshort{p4} underwent a major redesign\cite{p416} which simplified
the syntax and removed special-purpose language constructs in favour of more
general solutions\footnote{Specifically, features like counters and checksum
units were replaced by \texttt{extern}s, a universal construct for specifying
additional hardware capabilities not covered by the core \todo[inline]{word
here?}. The language shrank from over 70 to less than 40
keywords.\cite{p416:v1:spec:comparison}}. The redesigned language is known as
\pfs\cite{p416:v123:spec} and its specification has since received more
incremental updates.

%-------------------------------%
\section*{A Growing Community}
%-------------------------------%

\acrshort{p4} gave rise to a varied ecosystem of commercial offerings of both
hardware and software. It spawned several academic projects that investigated
its semantics\cite{doenges2021petr4}, allocation to heterogeneous
hardware\cite{sultana2021flightplan}, open-source network
testing\cite{antichi2014osnt}, and sketch-based
monitoring\cite{namkung2022sketchlib}, among many
others\cite{liatifis2023advancingp4survey}. The \acrfull{onf}
maintains\cite{p4onf} an open-source reference implementation of a
\pfs\footnote{Although primarily developed for \pfs since the revision of the
language, it is also capable of migrating P4\textsubscript{14} programs to \pfs
or directly compiling them.} compiler frontend and mid\-end, along with several
backends:

\begin{description}
	\item[p4c-bm2-ss] Targets a sample software switch for testing purposes.
	\item[p4c-dpdk] Targets the DPDK software switch (SWX) pipeline\cite{dpdkDPDKRelease}.
	\item[p4c-ebpf] Generates C code which can be compiled to eBPF
		and then loaded in the Linux kernel.
	\item[p4test] A source-to-source P4 translator for testing, learning compiler
		internals and debugging.
	\item[p4c-graphs] Generates visual representations of P4 programs.
	\item[p4c-ubpf] Generates eBPF code that runs in user-space.
	\item[p4tools] A platform for P4 test utilities, includes a test-case
		generator for P4 programs.
\end{description}

All of these components are open source. The frontend and mid\-end of the
reference compiler provide a foundation for hardware vendors to support
\acrshort{p4} in their products and serve the community in resolving
discrepancies between commercial compilers and the language specification.

Despite this growth, \acrshort{p4} has only mediocre support for real-time
feedback to the programmer -- the vast majority of open and commercial tools
rely on compiler output to provide semantic insight into a P4
program\cite{p4insight}, or, as evidenced by the open backends, are parts of the
compiler proper.\todo{make the case for real-time feedback and better error
reporting}

%-----------------------------------------%
\section*{Language Servers to the Rescue}
%-----------------------------------------%

Over the last decade\todo{poor opening? This should be in another section too},
\emph{language servers} became a popular architecture\cite{barros2022editing}
for providing semantically informed editing features in integrated developer
environments and lightweight source code editors alike. Although examples of
language server -like tools predate their standardization\cite{bour2018merlin},
a major milestone in their development was the introduction of the
\acrfull{lsp}. The support of \acrshort{lsp} in Visual Studio Code seeded an
ever-growing environment of cross-platform tooling for a variety of programming,
configuration, specification, and markup languages. Their success can be
attributed in part to the investment by Microsoft. However, \acrshort{lsp} as a
standard has much technical merit.

Before the introduction of a common communication interface between
language-specific tooling and a given source code editor, implementors had to
create and maintain extensions for any number of different environments, often
in several programming languages. For example, take an editor plugin providing
smart completion and "go to definition" features for the Python programming
language. Suppose the author aims to extend Vim, Emacs, and IntelliJ IDEA. They
would therefore have to reimplement the given functionality in vimscript, Emacs
Lisp, and a JVM-based language. Even though recent innovations in classical text
editors\todo{Neovim} have partly moved away from using their own
\acrshort{dsl}'s, the author is still burdened with three times the
implementation work, as every development environment \acrshort{api} is
different.

The solution to this problem is to split editor extensions according to the
client-server model. The major implementation work for domain-specific
functionality resides on the server side, whereas the client is only a
relatively thin wrapper that adapts a given editor's API. Implementors can build
many thin clients for different development environments that all connect to the
same \emph{language server}. \acrshort{lsp}'s raison d'être is support of this
model in Visual Studio Code.

Coden\-vy, Microsoft, and Red Hat collaborated on standardizing the protocol to
enable other tools to benefit from shared
tooling\cite{sdtimesCodenvyMicrosoft,infoworldMicrosoftbackedLanguage}.

% these comments make the section visible from the editor minimap
%-------------------%
%-------------------%
%-------------------%
%-------------------%
%-------------------%
%-------------------%
%-------------------%
%-------------------%
%-------------------%
%-------------------%
%-------------------%
\section{Outline}
%-------------------%
%-------------------%
%-------------------%
%-------------------%
%-------------------%
%-------------------%
%-------------------%
%-------------------%
%-------------------%
%-------------------%
%-------------------%
%-------------------%

\begin{enumerate}
	\item (intro) P4 information
	\item (intro) language servers, VS Code (screenshots, c++?)
	\item design: overview of the architecture, interfaces with the editor, what
		needs to be implemented
	\item subsections about problems that needed solving, but most should go into
		the design part!
	\item results, what it can do, open-source repository, performance metrics
		(aim for interactivity)
	\item conclusion, next steps
\end{enumerate}








\todo[inline]{authorship: link to GitHub, detail which non-critical parts were
implemented by a colleague}


\begin{lstlisting}[
	caption={~Useless code},
	label=list:8-6,
	captionpos=t,
	float,
	abovecaptionskip=-\medskipamount,
	belowcaptionskip=\medskipamount,
	language=C
]
	#include<stdio.h>
	#include<iostream>
	// A comment
	int main(void)
	{
		printf("Hello World\n");
		return 0;
	}
\end{lstlisting}

%%%%%%%%%%%%%%%%%%%%%%%%%%%%%%%%%
% alternative using package minted for source highlighting
% package minted requires execution with `-shell-escape'
% e.g., `xelatex -shell-escape ctufit-thesis.tex'
% \begin{listing}
% \caption{Zbytečný kód}\label{list:8-6}
% \begin{minted}{C}
%     #include<stdio.h>
%     #include<iostream>
%     // A comment
%     int main(void)
%     {
%         printf("Hello World\n");
%         return 0;
%     }
% \end{minted}
% \end{listing}
% %%%%%%%%%%%%%%%%%%%%%%%%%%%%%%%%%

\begin{table}\centering
\caption[Příklad tabulky]{~Typesetting math}\label{tab:mathematics}
\begin{tabular}{l|l|c|c}
	Typ	& Prostředí	&
		\LaTeX{}ovská zkratka	& \TeX{}ovská zkratka	\tabularnewline \hline
	Text	& \verb|math|	&
		\verb|\(...\)|	& \verb|$...$|	\tabularnewline \hline
	Displayed	& \verb|displaymath|	&
		\verb|\[...\]|	& \verb|$$...$$|	\tabularnewline
\end{tabular}
\end{table}
